\documentclass[a4paper]{article}
\usepackage{geometry}
\usepackage{graphicx}
\usepackage{natbib}
\usepackage{amsmath}
\usepackage{amssymb}
\usepackage{amsthm}
\usepackage{paralist}
\usepackage{epstopdf}
\usepackage{tabularx}
\usepackage{longtable}
\usepackage{multirow}
\usepackage{multicol}
\usepackage[hidelinks]{hyperref}
\usepackage{fancyvrb}
\usepackage{algorithm}
\usepackage{algorithmic}
\usepackage{float}
\usepackage{paralist}
\usepackage[svgname]{xcolor}
\usepackage{enumerate}
\usepackage{array}
\usepackage{times}
\usepackage{url}
\usepackage{fancyhdr}
\usepackage{comment}
\usepackage{environ}
\usepackage{times}
\usepackage{textcomp}
\usepackage{caption}
\usepackage{color}
\usepackage{xcolor}
\usepackage{forest}

\urlstyle{rm}

\setlength\parindent{0pt} % Removes all indentation from paragraphs
\theoremstyle{definition}
\newtheorem{definition}{Definition}[]
\newtheorem{conjecture}{Conjecture}[]
\newtheorem{example}{Example}[]
\newtheorem{theorem}{Theorem}[]
\newtheorem{lemma}{Lemma}
\newtheorem{proposition}{Proposition}
\newtheorem{corollary}{Corollary}

\floatname{algorithm}{Procedure}
\renewcommand{\algorithmicrequire}{\textbf{Input:}}
\renewcommand{\algorithmicensure}{\textbf{Output:}}
\newcommand{\abs}[1]{\lvert#1\rvert}
\newcommand{\norm}[1]{\lVert#1\rVert}
\newcommand{\RR}{\mathbb{R}}
\newcommand{\CC}{\mathbb{C}}
\newcommand{\Nat}{\mathbb{N}}
\newcommand{\br}[1]{\{#1\}}
\DeclareMathOperator*{\argmin}{arg\,min}
\DeclareMathOperator*{\argmax}{arg\,max}
\renewcommand{\qedsymbol}{$\blacksquare$}

\definecolor{dkgreen}{rgb}{0,0.6,0}
\definecolor{gray}{rgb}{0.5,0.5,0.5}
\definecolor{mauve}{rgb}{0.58,0,0.82}

\newcommand{\Var}{\mathrm{Var}}
\newcommand{\Cov}{\mathrm{Cov}}

\newcommand{\vc}[1]{\boldsymbol{#1}}
\newcommand{\xv}{\vc{x}}
\newcommand{\Sigmav}{\vc{\Sigma}}
\newcommand{\alphav}{\vc{\alpha}}
\newcommand{\muv}{\vc{\mu}}

\newcommand{\red}[1]{\textcolor{red}{#1}}

\def\x{\mathbf x}
\def\y{\mathbf y}
\def\w{\mathbf w}
\def\v{\mathbf v}
\def\E{\mathbb E}
\def\V{\mathbb V}

% TO SHOW SOLUTIONS, include following (else comment out):
\newenvironment{soln}{
    \leavevmode\color{blue}\ignorespaces
}{}


\hypersetup{
%    colorlinks,
    linkcolor={red!50!black},
    citecolor={blue!50!black},
    urlcolor={blue!80!black}
}

\geometry{
  top=1in,            % <-- you want to adjust this
  inner=1in,
  outer=1in,
  bottom=1in,
  headheight=3em,       % <-- and this
  headsep=2em,          % <-- and this
  footskip=3em,
}


\pagestyle{fancyplain}
\lhead{\fancyplain{}{Homework 2}}
\rhead{\fancyplain{}{CS 760 Machine Learning}}
\cfoot{\thepage}

\title{\textsc{Homework 2}} % Title

%%% NOTE:  Replace 'NAME HERE' etc., and delete any "\red{}" wrappers (so it won't show up as red)

\author{Dario Placencio\\
\texttt{907 284 6018}}

\date{}

\begin{document}

\maketitle 

\textbf{Instructions:} 
Use this latex file as a template to develop your homework. Submit your homework on time as a single pdf file to Canvas. Please wrap your code and upload to a public GitHub repo, then attach the link below the instructions so that we can access it. You can choose any programming language (i.e. python, R, or MATLAB), as long as you implement the algorithm from scratch (e.g. do not use sklearn on questions 1 to 7 in section 2). Please check Piazza for updates about the homework.

\begin{itemize}
  \item The Jupyter Noteebok userd for this homework can be found on this link: \url{https://github.com/placenciohid/ECE760-Homework/blob/205d92d9c0aaa05ab6ed142faf0d3fecb14a08c5/Homework%202/Homework%202%20-%20Dario%20Placencio.ipynb}
\end{itemize}

\section{A Simplified Decision Tree}
You are to implement a decision-tree learner for classification.
To simplify your work, this will not be a general purpose decision tree.  Instead, your program can assume that
\begin{itemize}
\item each item has two continuous features $\x \in \RR^2$
\item the class label is binary and encoded as $y \in \{0,1\}$
\item data files are in plaintext with one labeled item per line, separated by whitespace:
$$x_{11} \quad x_{12} \quad y_1$$
$$...$$
$$x_{n1} \quad x_{n2} \quad y_n$$
\end{itemize}

Your program should implement a decision tree learner according to the following guidelines:
\begin{itemize}
\item Candidate splits $(j,c)$ for numeric features should use a threshold $c$ in feature dimension $j$ in the form of $x_{j}\ge c$.
\item $c$ should be on values of that dimension present in the training data; i.e. the threshold is on training points, not in between training points. You may enumerate all features, and for each feature, use all possible values for that dimension.
\item You may skip those candidate splits with zero split information (i.e. the entropy of the split), and continue the enumeration.
\item The left branch of such a split is the ``then'' branch, and the right branch is ``else''.
\item Splits should be chosen using information gain ratio. If there is a tie you may break it arbitrarily.
\item The stopping criteria (for making a node into a leaf) are that 
	\begin{itemize}
	\item the node is empty, or
	\item all splits have zero gain ratio (if the entropy of the split is non-zero), or
	\item the entropy of any candidates split is zero
	\end{itemize}
\item To simplify, whenever there is no majority class in a leaf, let it predict $y=1$.
\end{itemize}

\section{Questions}
\begin{enumerate}
\item (Our algorithm stops at pure labels) [10 pts] If a node is not empty but contains training items with the same label, why is it guaranteed to become a leaf?  Explain. You may assume that the feature values of these items are not all the same. \\

If a node contains training items with the same label, then the entropy of the node is \(0\). This is because entropy measures the amount of uncertainty or randomness in a set. If all the labels are the same, there is no uncertainty. The formula for entropy is:
\[ \text{entropy}(S) = -p_+ \log_2(p_+) - p_- \log_2(p_-) \]

Where:

\begin{itemize}
    \item \( p_+ \) is the proportion of positive examples in \( S \)
    \item \( p_- \) is the proportion of negative examples in \( S \)
\end{itemize}

In the case where all examples have the same label, one of the proportions (either \( p_+ \) or \( p_- \)) will be \(1\), and the other will be \(0\). This results in an entropy of \(0\).
Given our decision tree's stopping criterion, if the entropy of a node is \(0\), we make that node a leaf. This is because there is no benefit to splitting a node with zero entropy further: we already have a perfectly accurate classification for the training items in that node.

\item (Our algorithm is greedy)  [10 pts] Handcraft a small training set where both classes are present but the algorithm refuses to split; instead it makes the root a leaf and stop;
Importantly, if we were to manually force a split, the algorithm will happily continue splitting the data set further and produce a deeper tree with zero training error.
You should (1) plot your training set, (2) explain why.  Hint: you don't need more than a handful of items. \\

In this specially created dataset, we have two groups labeled as 0 and 1, but if we look at the values of our features, there's no easy way to separate these groups with just one question (split). The data points are all mixed up, and there's no obvious first question we can ask to neatly separate the groups.\\

The Decision Tree algorithm wants to ask questions to figure things out. However, when it looks at this data, it can't find a good first question to start with. So, it stops at the very beginning, and will make the root node a leaf.\\

If we manually force a split at any feature, the algorithm will continue splitting the data further, producing a deeper tree with zero training error, but it will still start as a leaf node if no informative split is found at the root, like on this case.\\

\item (Information gain ratio exercise)  [10 pts] Use the training set Druns.txt.  For the root node, list all candidate cuts and their information gain ratio. If the entropy of the candidate split is zero, please list its mutual information (i.e. information gain). Hint: to get $\log_2(x)$ when your programming language may be using a different base, use \verb|log(x)/log(2)|. Also, please follow the split rule in the first section. \\

For the root node, considering the candidate cuts on the dataset \texttt{Druns.txt}, we can observe the following Information Gain Ratios (GR) and Information Gains (IG) for each threshold value:

\textbf{For Feature 1 (x1):}
\begin{itemize}
    \item Threshold \(0.0\): Mutual Information = \(0.0\)
    \item Threshold \(0.1\): Gain Ratio = \(0.1005\)
\end{itemize}

\textbf{For Feature 2 (x2):}
\begin{itemize}
    \item Threshold \(-2.0\): Mutual Information = \(0.0\)
    \item Threshold \(-1.0\): Gain Ratio = \(0.1005\)
    \item Threshold \(0.0\): Gain Ratio = \(0.0560\)
    \item Threshold \(1.0\): Gain Ratio = \(0.0058\)
    \item Threshold \(2.0\): Gain Ratio = \(0.0011\)
    \item Threshold \(3.0\): Gain Ratio = \(0.0164\)
    \item Threshold \(4.0\): Gain Ratio = \(0.0497\)
    \item Threshold \(5.0\): Gain Ratio = \(0.1112\)
    \item Threshold \(6.0\): Gain Ratio = \(0.2361\)
    \item Threshold \(7.0\): Gain Ratio = \(0.0560\)
    \item Threshold \(8.0\): Gain Ratio = \(0.4302\)
\end{itemize}

There are two unique values for Feature 0 (x1) that were tested as potential splits. One of the splits (Threshold 0.0) had an entropy of 0 (meaning it was a perfect split for the data it was given) and thus its mutual information is listed.\\

For feature 1 (x2) had many more unique values and hence more potential splits were tested. While most thresholds provided some amount of gain, the one at Threshold 8.0 offered the highest gain ratio of 0.4302, indicating that it might be the most informative split if you were to select the first split for a decision tree based purely on gain ratio.\\

Similarly, for Feature 1 with Threshold -2.0, the entropy of the candidate split is 0, so mutual information is displayed.\\

\item (The king of interpretability)  [10 pts] Decision tree is not the most accurate classifier in general.  However, it persists.  This is largely due to its rumored interpretability: a data scientist can easily explain a tree to a non-data scientist.  Build a tree from D3leaves.txt.  Then manually convert your tree to a set of logic rules.  Show the tree\footnote{When we say show the tree, we mean either the standard computer science tree view, or some crude plaintext representation of the tree -- as long as you explain the format.  When we say visualize the tree, we mean a plot in the 2D $\x$ space that shows how the tree will classify any points.} and the rules. \\

\begin{forest}
  for tree={
    grow=east,
    parent anchor=east,
    child anchor=west,
    anchor=west,
    calign=edge midpoint,
    inner sep=2pt,
    outer sep=1pt,
    edge={line width=0.7pt},
    if n children=0{tier=terminus}{}
  }
  [\(x_1 \geq 10.0\)
    [1]
    [\(x_2 \geq 3.0\)
      [1]
      [0]
    ]
  ]
\end{forest}

Rules of the Tree:

\begin{enumerate}
  \item IF $(x_1 \geq 10.0)$ THEN class = 1
  \item IF $(x_1 < 10.0)$ AND $(x_2 \geq 3.0)$ THEN class = 1
  \item IF $(x_1 < 10.0)$ AND $(x_2 < 3.0)$ THEN class = 0
\end{enumerate}


\item (Or is it?)  [10 pts] For this question only, make sure you DO NOT VISUALIZE the data sets or plot your tree's decision boundary in the 2D $\x$ space.  If your code does that, turn it off before proceeding.  This is because you want to see your own reaction when trying to interpret a tree.  You will get points no matter what your interpretation is.
And we will ask you to visualize them in the next question anyway.
  \begin{itemize}
  
  \item Build a decision tree on D1.txt.  Show it to us in any format (e.g. could be a standard binary tree with nodes and arrows, and denote the rule at each leaf node; or as simple as plaintext output where each line represents a node with appropriate line number pointers to child nodes; whatever is convenient for you). Again, do not visualize the data set or the tree in the $\x$ input space.  In real tasks you will not be able to visualize the whole high dimensional input space anyway, so we don't want you to ``cheat'' here. 
  
  \begin{forest}
    [\(x^2 \geq 0.201829\)
      [1]
      [0]
    ]
  \end{forest}

  \item Look at your tree in the above format (remember, you should not visualize the 2D dataset or your tree's decision boundary) and try to interpret the decision boundary in human understandable English. 

From the decision tree built on D1.txt, only a single split is made, with the rule being:

% reset enumerate counter
\setcounter{enumi}{0}
\begin{enumerate}
  \item If X2 is greater than or equal to 0.201829, then the class is 1.
  \item If X2 is less than 0.201829, then the class is 0.
\end{enumerate}
  
Regarding the decision boundary is difficult to interpret the total lack of knowledge of the magnitude of the features. However, we can say that the decision boundary is a vertical line at X2 = 0.201829.

  \item Build a decision tree on D2.txt.  Show it to us. 

  \item Try to interpret your D2 decision tree. Is it easy or possible to do so without visualization? \\
  
  \end{itemize}

  \begin{forest}
    for tree={
      grow'=0,
      parent anchor=south,
      child anchor=north,
      anchor=north,
      align=center,
      l sep=2cm,
      edge={-latex, line width=0.5pt}, % Adjust the line width as needed
    }
    [\(x_1 \geq 0.533076\)
      [\(x_2 \geq 0.228007\)
        [1]
        [0]
      ]
      [\(x_2 \geq 0.424906\)
        [\(x_2 \geq 0.708127\)
          [1]
          [0]
        ]
        [1]
      ]
    ]
  \end{forest}

  This is only a small portion of the tree, but it's enough to see the complexity of the tree. In the previous case, the tree was very simple, with only one split being made. The rest of the tree can be found on the code pages.\\

  So, this case is the totally of oposit, with many splits being made. From here I could suggest that the tree is more complex, and sensitive to the features, hence the need for more splits. However, I'm not sure if this is the case, without visualizing the data, so it's difficult to interpret the decision tree.

  \item (Hypothesis space)  [10 pts] For D1.txt and D2.txt, do the following separately:
  \begin{itemize}
  
  \item Produce a scatter plot of the data set.

  \item Visualize your decision tree's decision boundary (or decision region, or some other ways to clearly visualize how your decision tree will make decisions in the feature space).

  \item Then discuss why the size of your decision trees on D1 and D2 differ.  Relate this to the hypothesis space of our decision tree algorithm. \\

  \end{itemize}

It can be observed the difference in complexity for D2 compared to D1, the distribution of the labels on D1 makes the tree quite simple, considering the strong effect of X2 on the label. On the case of D2 is the oposit, the distribution of the labels is more complex, and the effect of X1 and X2 is not as strong as in D1, hence the need for more splits.  

\item (Learning curve)  [20 pts] We provide a data set Dbig.txt with 10000 labeled items.  Caution: Dbig.txt is sorted.
  \begin{itemize}
  
  \item You will randomly split Dbig.txt into a candidate training set of 8192 items and a test set (the rest).  Do this by generating a random permutation, and split at 8192.
  
  \item Generate a sequence of five nested training sets $D_{32} \subset D_{128} \subset D_{512} \subset D_{2048} \subset D_{8192}$ from the candidate training set.  The subscript $n$ in $D_n$ denotes training set size.  The easiest way is to take the first $n$ items from the (same) permutation above.  This sequence simulates the real world situation where you obtain more and more training data.
  
  \item For each $D_n$ above, train a decision tree.  Measure its test set error $err_n$.  Show three things in your answer: (1) List $n$, number of nodes in that tree, $err_n$. (2) Plot $n$ vs. $err_n$.  This is known as a learning curve (a single plot). (3) Visualize your decision trees' decision boundary (five plots). \\
  \end{itemize}
  
\end{enumerate}

\section{sklearn [10 pts]}
Learn to use sklearn (\url{https://scikit-learn.org/stable/}).
Use sklearn.tree.DecisionTreeClassifier to produce trees for datasets $D_{32}, D_{128}, D_{512}, D_{2048}, D_{8192}$.  Show two things in your answer: (1) List $n$, number of nodes in that tree, $err_n$. (2) Plot $n$ vs. $err_n$.

\section{Lagrange Interpolation [10 pts]}
Fix some interval $[a, b]$ and sample $n = 100$ points $x$ from this interval uniformly. Use these to build a training set consisting of $n$ pairs $(x, y)$ by setting function $y = sin(x)$. \\

Build a model $f$ by using Lagrange interpolation, check more details in \url{https://en.wikipedia.org/wiki/Lagrange_polynomial} and \url{https://docs.scipy.org/doc/scipy/reference/generated/scipy.interpolate.lagrange.html}. \\

Generate a test set using the same distribution as your test set. Compute and report the resulting model’s train and test error. What do you observe?\\

It can be observed that errors for train and test are relatively similar, but both have a considerable magnitude that might be the nature of the interpolation method, specially considering the big number of points used for the interpolation.\\

Repeat the experiment with zero-mean Gaussian noise $\epsilon$ added to $x$. Vary the standard deviation for $\epsilon$ and report your findings.\\

It can be observed that when Gaussian noise is introduced to the training data for Lagrange interpolation, the model's performance initially improves with moderate noise levels, likely due to noise acting as a regularizer and preventing overfitting. However, there's an optimal noise threshold (around a standard deviation of 10) beyond which the performance starts to decline. Essentially, a certain amount of noise can enhance the model's generalization, but excessive noise deteriorates its capability.

\bibliographystyle{apalike}
\end{document}
