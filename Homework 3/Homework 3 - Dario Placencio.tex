\documentclass[11pt]{article}

    \usepackage[breakable]{tcolorbox}
    \usepackage{parskip} % Stop auto-indenting (to mimic markdown behaviour)
    

    % Basic figure setup, for now with no caption control since it's done
    % automatically by Pandoc (which extracts ![](path) syntax from Markdown).
    \usepackage{graphicx}
    % Maintain compatibility with old templates. Remove in nbconvert 6.0
    \let\Oldincludegraphics\includegraphics
    % Ensure that by default, figures have no caption (until we provide a
    % proper Figure object with a Caption API and a way to capture that
    % in the conversion process - todo).
    \usepackage{caption}
    \DeclareCaptionFormat{nocaption}{}
    \captionsetup{format=nocaption,aboveskip=0pt,belowskip=0pt}

    \usepackage{float}
    \floatplacement{figure}{H} % forces figures to be placed at the correct location
    \usepackage{xcolor} % Allow colors to be defined
    \usepackage{enumerate} % Needed for markdown enumerations to work
    \usepackage{geometry} % Used to adjust the document margins
    \usepackage{amsmath} % Equations
    \usepackage{amssymb} % Equations
    \usepackage{textcomp} % defines textquotesingle
    % Hack from http://tex.stackexchange.com/a/47451/13684:
    \AtBeginDocument{%
        \def\PYZsq{\textquotesingle}% Upright quotes in Pygmentized code
    }
    \usepackage{upquote} % Upright quotes for verbatim code
    \usepackage{eurosym} % defines \euro

    \usepackage{iftex}
    \ifPDFTeX
        \usepackage[T1]{fontenc}
        \IfFileExists{alphabeta.sty}{
              \usepackage{alphabeta}
          }{
              \usepackage[mathletters]{ucs}
              \usepackage[utf8x]{inputenc}
          }
    \else
        \usepackage{fontspec}
        \usepackage{unicode-math}
    \fi

    \usepackage{fancyvrb} % verbatim replacement that allows latex
    \usepackage{grffile} % extends the file name processing of package graphics
                         % to support a larger range
    \makeatletter % fix for old versions of grffile with XeLaTeX
    \@ifpackagelater{grffile}{2019/11/01}
    {
      % Do nothing on new versions
    }
    {
      \def\Gread@@xetex#1{%
        \IfFileExists{"\Gin@base".bb}%
        {\Gread@eps{\Gin@base.bb}}%
        {\Gread@@xetex@aux#1}%
      }
    }
    \makeatother
    \usepackage[Export]{adjustbox} % Used to constrain images to a maximum size
    \adjustboxset{max size={0.9\linewidth}{0.9\paperheight}}

    % The hyperref package gives us a pdf with properly built
    % internal navigation ('pdf bookmarks' for the table of contents,
    % internal cross-reference links, web links for URLs, etc.)
    \usepackage{hyperref}
    % The default LaTeX title has an obnoxious amount of whitespace. By default,
    % titling removes some of it. It also provides customization options.
    \usepackage{titling}
    \usepackage{longtable} % longtable support required by pandoc >1.10
    \usepackage{booktabs}  % table support for pandoc > 1.12.2
    \usepackage{array}     % table support for pandoc >= 2.11.3
    \usepackage{calc}      % table minipage width calculation for pandoc >= 2.11.1
    \usepackage[inline]{enumitem} % IRkernel/repr support (it uses the enumerate* environment)
    \usepackage[normalem]{ulem} % ulem is needed to support strikethroughs (\sout)
                                % normalem makes italics be italics, not underlines
    \usepackage{mathrsfs}
    

    
    % Colors for the hyperref package
    \definecolor{urlcolor}{rgb}{0,.145,.698}
    \definecolor{linkcolor}{rgb}{.71,0.21,0.01}
    \definecolor{citecolor}{rgb}{.12,.54,.11}

    % ANSI colors
    \definecolor{ansi-black}{HTML}{3E424D}
    \definecolor{ansi-black-intense}{HTML}{282C36}
    \definecolor{ansi-red}{HTML}{E75C58}
    \definecolor{ansi-red-intense}{HTML}{B22B31}
    \definecolor{ansi-green}{HTML}{00A250}
    \definecolor{ansi-green-intense}{HTML}{007427}
    \definecolor{ansi-yellow}{HTML}{DDB62B}
    \definecolor{ansi-yellow-intense}{HTML}{B27D12}
    \definecolor{ansi-blue}{HTML}{208FFB}
    \definecolor{ansi-blue-intense}{HTML}{0065CA}
    \definecolor{ansi-magenta}{HTML}{D160C4}
    \definecolor{ansi-magenta-intense}{HTML}{A03196}
    \definecolor{ansi-cyan}{HTML}{60C6C8}
    \definecolor{ansi-cyan-intense}{HTML}{258F8F}
    \definecolor{ansi-white}{HTML}{C5C1B4}
    \definecolor{ansi-white-intense}{HTML}{A1A6B2}
    \definecolor{ansi-default-inverse-fg}{HTML}{FFFFFF}
    \definecolor{ansi-default-inverse-bg}{HTML}{000000}

    % common color for the border for error outputs.
    \definecolor{outerrorbackground}{HTML}{FFDFDF}

    % commands and environments needed by pandoc snippets
    % extracted from the output of `pandoc -s`
    \providecommand{\tightlist}{%
      \setlength{\itemsep}{0pt}\setlength{\parskip}{0pt}}
    \DefineVerbatimEnvironment{Highlighting}{Verbatim}{commandchars=\\\{\}}
    % Add ',fontsize=\small' for more characters per line
    \newenvironment{Shaded}{}{}
    \newcommand{\KeywordTok}[1]{\textcolor[rgb]{0.00,0.44,0.13}{\textbf{{#1}}}}
    \newcommand{\DataTypeTok}[1]{\textcolor[rgb]{0.56,0.13,0.00}{{#1}}}
    \newcommand{\DecValTok}[1]{\textcolor[rgb]{0.25,0.63,0.44}{{#1}}}
    \newcommand{\BaseNTok}[1]{\textcolor[rgb]{0.25,0.63,0.44}{{#1}}}
    \newcommand{\FloatTok}[1]{\textcolor[rgb]{0.25,0.63,0.44}{{#1}}}
    \newcommand{\CharTok}[1]{\textcolor[rgb]{0.25,0.44,0.63}{{#1}}}
    \newcommand{\StringTok}[1]{\textcolor[rgb]{0.25,0.44,0.63}{{#1}}}
    \newcommand{\CommentTok}[1]{\textcolor[rgb]{0.38,0.63,0.69}{\textit{{#1}}}}
    \newcommand{\OtherTok}[1]{\textcolor[rgb]{0.00,0.44,0.13}{{#1}}}
    \newcommand{\AlertTok}[1]{\textcolor[rgb]{1.00,0.00,0.00}{\textbf{{#1}}}}
    \newcommand{\FunctionTok}[1]{\textcolor[rgb]{0.02,0.16,0.49}{{#1}}}
    \newcommand{\RegionMarkerTok}[1]{{#1}}
    \newcommand{\ErrorTok}[1]{\textcolor[rgb]{1.00,0.00,0.00}{\textbf{{#1}}}}
    \newcommand{\NormalTok}[1]{{#1}}

    % Additional commands for more recent versions of Pandoc
    \newcommand{\ConstantTok}[1]{\textcolor[rgb]{0.53,0.00,0.00}{{#1}}}
    \newcommand{\SpecialCharTok}[1]{\textcolor[rgb]{0.25,0.44,0.63}{{#1}}}
    \newcommand{\VerbatimStringTok}[1]{\textcolor[rgb]{0.25,0.44,0.63}{{#1}}}
    \newcommand{\SpecialStringTok}[1]{\textcolor[rgb]{0.73,0.40,0.53}{{#1}}}
    \newcommand{\ImportTok}[1]{{#1}}
    \newcommand{\DocumentationTok}[1]{\textcolor[rgb]{0.73,0.13,0.13}{\textit{{#1}}}}
    \newcommand{\AnnotationTok}[1]{\textcolor[rgb]{0.38,0.63,0.69}{\textbf{\textit{{#1}}}}}
    \newcommand{\CommentVarTok}[1]{\textcolor[rgb]{0.38,0.63,0.69}{\textbf{\textit{{#1}}}}}
    \newcommand{\VariableTok}[1]{\textcolor[rgb]{0.10,0.09,0.49}{{#1}}}
    \newcommand{\ControlFlowTok}[1]{\textcolor[rgb]{0.00,0.44,0.13}{\textbf{{#1}}}}
    \newcommand{\OperatorTok}[1]{\textcolor[rgb]{0.40,0.40,0.40}{{#1}}}
    \newcommand{\BuiltInTok}[1]{{#1}}
    \newcommand{\ExtensionTok}[1]{{#1}}
    \newcommand{\PreprocessorTok}[1]{\textcolor[rgb]{0.74,0.48,0.00}{{#1}}}
    \newcommand{\AttributeTok}[1]{\textcolor[rgb]{0.49,0.56,0.16}{{#1}}}
    \newcommand{\InformationTok}[1]{\textcolor[rgb]{0.38,0.63,0.69}{\textbf{\textit{{#1}}}}}
    \newcommand{\WarningTok}[1]{\textcolor[rgb]{0.38,0.63,0.69}{\textbf{\textit{{#1}}}}}


    % Define a nice break command that doesn't care if a line doesn't already
    % exist.
    \def\br{\hspace*{\fill} \\* }
    % Math Jax compatibility definitions
    \def\gt{>}
    \def\lt{<}
    \let\Oldtex\TeX
    \let\Oldlatex\LaTeX
    \renewcommand{\TeX}{\textrm{\Oldtex}}
    \renewcommand{\LaTeX}{\textrm{\Oldlatex}}
    % Document parameters
    % Document title
    \title{Homework 3 - Dario Placencio}
    
    
    
    
    
    
    
% Pygments definitions
\makeatletter
\def\PY@reset{\let\PY@it=\relax \let\PY@bf=\relax%
    \let\PY@ul=\relax \let\PY@tc=\relax%
    \let\PY@bc=\relax \let\PY@ff=\relax}
\def\PY@tok#1{\csname PY@tok@#1\endcsname}
\def\PY@toks#1+{\ifx\relax#1\empty\else%
    \PY@tok{#1}\expandafter\PY@toks\fi}
\def\PY@do#1{\PY@bc{\PY@tc{\PY@ul{%
    \PY@it{\PY@bf{\PY@ff{#1}}}}}}}
\def\PY#1#2{\PY@reset\PY@toks#1+\relax+\PY@do{#2}}

\@namedef{PY@tok@w}{\def\PY@tc##1{\textcolor[rgb]{0.73,0.73,0.73}{##1}}}
\@namedef{PY@tok@c}{\let\PY@it=\textit\def\PY@tc##1{\textcolor[rgb]{0.24,0.48,0.48}{##1}}}
\@namedef{PY@tok@cp}{\def\PY@tc##1{\textcolor[rgb]{0.61,0.40,0.00}{##1}}}
\@namedef{PY@tok@k}{\let\PY@bf=\textbf\def\PY@tc##1{\textcolor[rgb]{0.00,0.50,0.00}{##1}}}
\@namedef{PY@tok@kp}{\def\PY@tc##1{\textcolor[rgb]{0.00,0.50,0.00}{##1}}}
\@namedef{PY@tok@kt}{\def\PY@tc##1{\textcolor[rgb]{0.69,0.00,0.25}{##1}}}
\@namedef{PY@tok@o}{\def\PY@tc##1{\textcolor[rgb]{0.40,0.40,0.40}{##1}}}
\@namedef{PY@tok@ow}{\let\PY@bf=\textbf\def\PY@tc##1{\textcolor[rgb]{0.67,0.13,1.00}{##1}}}
\@namedef{PY@tok@nb}{\def\PY@tc##1{\textcolor[rgb]{0.00,0.50,0.00}{##1}}}
\@namedef{PY@tok@nf}{\def\PY@tc##1{\textcolor[rgb]{0.00,0.00,1.00}{##1}}}
\@namedef{PY@tok@nc}{\let\PY@bf=\textbf\def\PY@tc##1{\textcolor[rgb]{0.00,0.00,1.00}{##1}}}
\@namedef{PY@tok@nn}{\let\PY@bf=\textbf\def\PY@tc##1{\textcolor[rgb]{0.00,0.00,1.00}{##1}}}
\@namedef{PY@tok@ne}{\let\PY@bf=\textbf\def\PY@tc##1{\textcolor[rgb]{0.80,0.25,0.22}{##1}}}
\@namedef{PY@tok@nv}{\def\PY@tc##1{\textcolor[rgb]{0.10,0.09,0.49}{##1}}}
\@namedef{PY@tok@no}{\def\PY@tc##1{\textcolor[rgb]{0.53,0.00,0.00}{##1}}}
\@namedef{PY@tok@nl}{\def\PY@tc##1{\textcolor[rgb]{0.46,0.46,0.00}{##1}}}
\@namedef{PY@tok@ni}{\let\PY@bf=\textbf\def\PY@tc##1{\textcolor[rgb]{0.44,0.44,0.44}{##1}}}
\@namedef{PY@tok@na}{\def\PY@tc##1{\textcolor[rgb]{0.41,0.47,0.13}{##1}}}
\@namedef{PY@tok@nt}{\let\PY@bf=\textbf\def\PY@tc##1{\textcolor[rgb]{0.00,0.50,0.00}{##1}}}
\@namedef{PY@tok@nd}{\def\PY@tc##1{\textcolor[rgb]{0.67,0.13,1.00}{##1}}}
\@namedef{PY@tok@s}{\def\PY@tc##1{\textcolor[rgb]{0.73,0.13,0.13}{##1}}}
\@namedef{PY@tok@sd}{\let\PY@it=\textit\def\PY@tc##1{\textcolor[rgb]{0.73,0.13,0.13}{##1}}}
\@namedef{PY@tok@si}{\let\PY@bf=\textbf\def\PY@tc##1{\textcolor[rgb]{0.64,0.35,0.47}{##1}}}
\@namedef{PY@tok@se}{\let\PY@bf=\textbf\def\PY@tc##1{\textcolor[rgb]{0.67,0.36,0.12}{##1}}}
\@namedef{PY@tok@sr}{\def\PY@tc##1{\textcolor[rgb]{0.64,0.35,0.47}{##1}}}
\@namedef{PY@tok@ss}{\def\PY@tc##1{\textcolor[rgb]{0.10,0.09,0.49}{##1}}}
\@namedef{PY@tok@sx}{\def\PY@tc##1{\textcolor[rgb]{0.00,0.50,0.00}{##1}}}
\@namedef{PY@tok@m}{\def\PY@tc##1{\textcolor[rgb]{0.40,0.40,0.40}{##1}}}
\@namedef{PY@tok@gh}{\let\PY@bf=\textbf\def\PY@tc##1{\textcolor[rgb]{0.00,0.00,0.50}{##1}}}
\@namedef{PY@tok@gu}{\let\PY@bf=\textbf\def\PY@tc##1{\textcolor[rgb]{0.50,0.00,0.50}{##1}}}
\@namedef{PY@tok@gd}{\def\PY@tc##1{\textcolor[rgb]{0.63,0.00,0.00}{##1}}}
\@namedef{PY@tok@gi}{\def\PY@tc##1{\textcolor[rgb]{0.00,0.52,0.00}{##1}}}
\@namedef{PY@tok@gr}{\def\PY@tc##1{\textcolor[rgb]{0.89,0.00,0.00}{##1}}}
\@namedef{PY@tok@ge}{\let\PY@it=\textit}
\@namedef{PY@tok@gs}{\let\PY@bf=\textbf}
\@namedef{PY@tok@ges}{\let\PY@bf=\textbf\let\PY@it=\textit}
\@namedef{PY@tok@gp}{\let\PY@bf=\textbf\def\PY@tc##1{\textcolor[rgb]{0.00,0.00,0.50}{##1}}}
\@namedef{PY@tok@go}{\def\PY@tc##1{\textcolor[rgb]{0.44,0.44,0.44}{##1}}}
\@namedef{PY@tok@gt}{\def\PY@tc##1{\textcolor[rgb]{0.00,0.27,0.87}{##1}}}
\@namedef{PY@tok@err}{\def\PY@bc##1{{\setlength{\fboxsep}{\string -\fboxrule}\fcolorbox[rgb]{1.00,0.00,0.00}{1,1,1}{\strut ##1}}}}
\@namedef{PY@tok@kc}{\let\PY@bf=\textbf\def\PY@tc##1{\textcolor[rgb]{0.00,0.50,0.00}{##1}}}
\@namedef{PY@tok@kd}{\let\PY@bf=\textbf\def\PY@tc##1{\textcolor[rgb]{0.00,0.50,0.00}{##1}}}
\@namedef{PY@tok@kn}{\let\PY@bf=\textbf\def\PY@tc##1{\textcolor[rgb]{0.00,0.50,0.00}{##1}}}
\@namedef{PY@tok@kr}{\let\PY@bf=\textbf\def\PY@tc##1{\textcolor[rgb]{0.00,0.50,0.00}{##1}}}
\@namedef{PY@tok@bp}{\def\PY@tc##1{\textcolor[rgb]{0.00,0.50,0.00}{##1}}}
\@namedef{PY@tok@fm}{\def\PY@tc##1{\textcolor[rgb]{0.00,0.00,1.00}{##1}}}
\@namedef{PY@tok@vc}{\def\PY@tc##1{\textcolor[rgb]{0.10,0.09,0.49}{##1}}}
\@namedef{PY@tok@vg}{\def\PY@tc##1{\textcolor[rgb]{0.10,0.09,0.49}{##1}}}
\@namedef{PY@tok@vi}{\def\PY@tc##1{\textcolor[rgb]{0.10,0.09,0.49}{##1}}}
\@namedef{PY@tok@vm}{\def\PY@tc##1{\textcolor[rgb]{0.10,0.09,0.49}{##1}}}
\@namedef{PY@tok@sa}{\def\PY@tc##1{\textcolor[rgb]{0.73,0.13,0.13}{##1}}}
\@namedef{PY@tok@sb}{\def\PY@tc##1{\textcolor[rgb]{0.73,0.13,0.13}{##1}}}
\@namedef{PY@tok@sc}{\def\PY@tc##1{\textcolor[rgb]{0.73,0.13,0.13}{##1}}}
\@namedef{PY@tok@dl}{\def\PY@tc##1{\textcolor[rgb]{0.73,0.13,0.13}{##1}}}
\@namedef{PY@tok@s2}{\def\PY@tc##1{\textcolor[rgb]{0.73,0.13,0.13}{##1}}}
\@namedef{PY@tok@sh}{\def\PY@tc##1{\textcolor[rgb]{0.73,0.13,0.13}{##1}}}
\@namedef{PY@tok@s1}{\def\PY@tc##1{\textcolor[rgb]{0.73,0.13,0.13}{##1}}}
\@namedef{PY@tok@mb}{\def\PY@tc##1{\textcolor[rgb]{0.40,0.40,0.40}{##1}}}
\@namedef{PY@tok@mf}{\def\PY@tc##1{\textcolor[rgb]{0.40,0.40,0.40}{##1}}}
\@namedef{PY@tok@mh}{\def\PY@tc##1{\textcolor[rgb]{0.40,0.40,0.40}{##1}}}
\@namedef{PY@tok@mi}{\def\PY@tc##1{\textcolor[rgb]{0.40,0.40,0.40}{##1}}}
\@namedef{PY@tok@il}{\def\PY@tc##1{\textcolor[rgb]{0.40,0.40,0.40}{##1}}}
\@namedef{PY@tok@mo}{\def\PY@tc##1{\textcolor[rgb]{0.40,0.40,0.40}{##1}}}
\@namedef{PY@tok@ch}{\let\PY@it=\textit\def\PY@tc##1{\textcolor[rgb]{0.24,0.48,0.48}{##1}}}
\@namedef{PY@tok@cm}{\let\PY@it=\textit\def\PY@tc##1{\textcolor[rgb]{0.24,0.48,0.48}{##1}}}
\@namedef{PY@tok@cpf}{\let\PY@it=\textit\def\PY@tc##1{\textcolor[rgb]{0.24,0.48,0.48}{##1}}}
\@namedef{PY@tok@c1}{\let\PY@it=\textit\def\PY@tc##1{\textcolor[rgb]{0.24,0.48,0.48}{##1}}}
\@namedef{PY@tok@cs}{\let\PY@it=\textit\def\PY@tc##1{\textcolor[rgb]{0.24,0.48,0.48}{##1}}}

\def\PYZbs{\char`\\}
\def\PYZus{\char`\_}
\def\PYZob{\char`\{}
\def\PYZcb{\char`\}}
\def\PYZca{\char`\^}
\def\PYZam{\char`\&}
\def\PYZlt{\char`\<}
\def\PYZgt{\char`\>}
\def\PYZsh{\char`\#}
\def\PYZpc{\char`\%}
\def\PYZdl{\char`\$}
\def\PYZhy{\char`\-}
\def\PYZsq{\char`\'}
\def\PYZdq{\char`\"}
\def\PYZti{\char`\~}
% for compatibility with earlier versions
\def\PYZat{@}
\def\PYZlb{[}
\def\PYZrb{]}
\makeatother


    % For linebreaks inside Verbatim environment from package fancyvrb.
    \makeatletter
        \newbox\Wrappedcontinuationbox
        \newbox\Wrappedvisiblespacebox
        \newcommand*\Wrappedvisiblespace {\textcolor{red}{\textvisiblespace}}
        \newcommand*\Wrappedcontinuationsymbol {\textcolor{red}{\llap{\tiny$\m@th\hookrightarrow$}}}
        \newcommand*\Wrappedcontinuationindent {3ex }
        \newcommand*\Wrappedafterbreak {\kern\Wrappedcontinuationindent\copy\Wrappedcontinuationbox}
        % Take advantage of the already applied Pygments mark-up to insert
        % potential linebreaks for TeX processing.
        %        {, <, #, %, $, ' and ": go to next line.
        %        _, }, ^, &, >, - and ~: stay at end of broken line.
        % Use of \textquotesingle for straight quote.
        \newcommand*\Wrappedbreaksatspecials {%
            \def\PYGZus{\discretionary{\char`\_}{\Wrappedafterbreak}{\char`\_}}%
            \def\PYGZob{\discretionary{}{\Wrappedafterbreak\char`\{}{\char`\{}}%
            \def\PYGZcb{\discretionary{\char`\}}{\Wrappedafterbreak}{\char`\}}}%
            \def\PYGZca{\discretionary{\char`\^}{\Wrappedafterbreak}{\char`\^}}%
            \def\PYGZam{\discretionary{\char`\&}{\Wrappedafterbreak}{\char`\&}}%
            \def\PYGZlt{\discretionary{}{\Wrappedafterbreak\char`\<}{\char`\<}}%
            \def\PYGZgt{\discretionary{\char`\>}{\Wrappedafterbreak}{\char`\>}}%
            \def\PYGZsh{\discretionary{}{\Wrappedafterbreak\char`\#}{\char`\#}}%
            \def\PYGZpc{\discretionary{}{\Wrappedafterbreak\char`\%}{\char`\%}}%
            \def\PYGZdl{\discretionary{}{\Wrappedafterbreak\char`\$}{\char`\$}}%
            \def\PYGZhy{\discretionary{\char`\-}{\Wrappedafterbreak}{\char`\-}}%
            \def\PYGZsq{\discretionary{}{\Wrappedafterbreak\textquotesingle}{\textquotesingle}}%
            \def\PYGZdq{\discretionary{}{\Wrappedafterbreak\char`\"}{\char`\"}}%
            \def\PYGZti{\discretionary{\char`\~}{\Wrappedafterbreak}{\char`\~}}%
        }
        % Some characters . , ; ? ! / are not pygmentized.
        % This macro makes them "active" and they will insert potential linebreaks
        \newcommand*\Wrappedbreaksatpunct {%
            \lccode`\~`\.\lowercase{\def~}{\discretionary{\hbox{\char`\.}}{\Wrappedafterbreak}{\hbox{\char`\.}}}%
            \lccode`\~`\,\lowercase{\def~}{\discretionary{\hbox{\char`\,}}{\Wrappedafterbreak}{\hbox{\char`\,}}}%
            \lccode`\~`\;\lowercase{\def~}{\discretionary{\hbox{\char`\;}}{\Wrappedafterbreak}{\hbox{\char`\;}}}%
            \lccode`\~`\:\lowercase{\def~}{\discretionary{\hbox{\char`\:}}{\Wrappedafterbreak}{\hbox{\char`\:}}}%
            \lccode`\~`\?\lowercase{\def~}{\discretionary{\hbox{\char`\?}}{\Wrappedafterbreak}{\hbox{\char`\?}}}%
            \lccode`\~`\!\lowercase{\def~}{\discretionary{\hbox{\char`\!}}{\Wrappedafterbreak}{\hbox{\char`\!}}}%
            \lccode`\~`\/\lowercase{\def~}{\discretionary{\hbox{\char`\/}}{\Wrappedafterbreak}{\hbox{\char`\/}}}%
            \catcode`\.\active
            \catcode`\,\active
            \catcode`\;\active
            \catcode`\:\active
            \catcode`\?\active
            \catcode`\!\active
            \catcode`\/\active
            \lccode`\~`\~
        }
    \makeatother

    \let\OriginalVerbatim=\Verbatim
    \makeatletter
    \renewcommand{\Verbatim}[1][1]{%
        %\parskip\z@skip
        \sbox\Wrappedcontinuationbox {\Wrappedcontinuationsymbol}%
        \sbox\Wrappedvisiblespacebox {\FV@SetupFont\Wrappedvisiblespace}%
        \def\FancyVerbFormatLine ##1{\hsize\linewidth
            \vtop{\raggedright\hyphenpenalty\z@\exhyphenpenalty\z@
                \doublehyphendemerits\z@\finalhyphendemerits\z@
                \strut ##1\strut}%
        }%
        % If the linebreak is at a space, the latter will be displayed as visible
        % space at end of first line, and a continuation symbol starts next line.
        % Stretch/shrink are however usually zero for typewriter font.
        \def\FV@Space {%
            \nobreak\hskip\z@ plus\fontdimen3\font minus\fontdimen4\font
            \discretionary{\copy\Wrappedvisiblespacebox}{\Wrappedafterbreak}
            {\kern\fontdimen2\font}%
        }%

        % Allow breaks at special characters using \PYG... macros.
        \Wrappedbreaksatspecials
        % Breaks at punctuation characters . , ; ? ! and / need catcode=\active
        \OriginalVerbatim[#1,codes*=\Wrappedbreaksatpunct]%
    }
    \makeatother

    % Exact colors from NB
    \definecolor{incolor}{HTML}{303F9F}
    \definecolor{outcolor}{HTML}{D84315}
    \definecolor{cellborder}{HTML}{CFCFCF}
    \definecolor{cellbackground}{HTML}{F7F7F7}

    % prompt
    \makeatletter
    \newcommand{\boxspacing}{\kern\kvtcb@left@rule\kern\kvtcb@boxsep}
    \makeatother
    \newcommand{\prompt}[4]{
        {\ttfamily\llap{{\color{#2}[#3]:\hspace{3pt}#4}}\vspace{-\baselineskip}}
    }
    

    
    % Prevent overflowing lines due to hard-to-break entities
    \sloppy
    % Setup hyperref package
    \hypersetup{
      breaklinks=true,  % so long urls are correctly broken across lines
      colorlinks=true,
      urlcolor=urlcolor,
      linkcolor=linkcolor,
      citecolor=citecolor,
      }
    % Slightly bigger margins than the latex defaults
    
    \geometry{verbose,tmargin=1in,bmargin=1in,lmargin=1in,rmargin=1in}
    
    

\begin{document}
    
    \maketitle
    
    

    
    \hypertarget{homework-3---dario-placencio}{%
\section{Homework 3 - Dario
Placencio}\label{homework-3---dario-placencio}}

    \hypertarget{questions}{%
\section{1 Questions}\label{questions}}

    \begin{tcolorbox}[breakable, size=fbox, boxrule=1pt, pad at break*=1mm,colback=cellbackground, colframe=cellborder]
\prompt{In}{incolor}{1}{\boxspacing}
\begin{Verbatim}[commandchars=\\\{\}]
\PY{c+c1}{\PYZsh{} import the required libraries}
\PY{k+kn}{import} \PY{n+nn}{numpy} \PY{k}{as} \PY{n+nn}{np}
\PY{k+kn}{import} \PY{n+nn}{pandas} \PY{k}{as} \PY{n+nn}{pd}
\PY{k+kn}{import} \PY{n+nn}{matplotlib}\PY{n+nn}{.}\PY{n+nn}{pyplot} \PY{k}{as} \PY{n+nn}{plt}
\PY{k+kn}{import} \PY{n+nn}{seaborn} \PY{k}{as} \PY{n+nn}{sns}
\end{Verbatim}
\end{tcolorbox}

    \begin{enumerate}
\def\labelenumi{\arabic{enumi}.}
\setcounter{enumi}{1}
\tightlist
\item
  (6 pts) The table below provides a training data set containing six
  observations, three predictors, and one qualitative response variable.
\end{enumerate}

\[
\begin{array}{cccc}
\hline
X_{1} & X_{2} & X_{3} & Y \\
\hline
0 & 3 & 0 & \text{Red} \\
2 & 0 & 0 & \text{Red} \\
0 & 1 & 3 & \text{Red} \\
0 & 1 & 2 & \text{Green} \\
-1 & 0 & 1 & \text{Green} \\
1 & 1 & 1 & \text{Red} \\
\hline
\end{array}
\]

Suppose we wish to use this data set to make a prediction for \(Y\) when
\(X_{1} = X_{2} = X_{3} = 0\) using K-nearest neighbors.

    \begin{tcolorbox}[breakable, size=fbox, boxrule=1pt, pad at break*=1mm,colback=cellbackground, colframe=cellborder]
\prompt{In}{incolor}{8}{\boxspacing}
\begin{Verbatim}[commandchars=\\\{\}]
\PY{c+c1}{\PYZsh{} Create the dataframe}

\PY{n}{table\PYZus{}data} \PY{o}{=} \PY{p}{\PYZob{}}\PY{l+s+s1}{\PYZsq{}}\PY{l+s+s1}{Y}\PY{l+s+s1}{\PYZsq{}}\PY{p}{:} \PY{p}{[}\PY{l+s+s2}{\PYZdq{}}\PY{l+s+s2}{Red}\PY{l+s+s2}{\PYZdq{}}\PY{p}{,} \PY{l+s+s2}{\PYZdq{}}\PY{l+s+s2}{Red}\PY{l+s+s2}{\PYZdq{}}\PY{p}{,} \PY{l+s+s2}{\PYZdq{}}\PY{l+s+s2}{Red}\PY{l+s+s2}{\PYZdq{}}\PY{p}{,} \PY{l+s+s2}{\PYZdq{}}\PY{l+s+s2}{Green}\PY{l+s+s2}{\PYZdq{}}\PY{p}{,} \PY{l+s+s2}{\PYZdq{}}\PY{l+s+s2}{Green}\PY{l+s+s2}{\PYZdq{}}\PY{p}{,} \PY{l+s+s2}{\PYZdq{}}\PY{l+s+s2}{Red}\PY{l+s+s2}{\PYZdq{}}\PY{p}{]}\PY{p}{,} 
                     \PY{l+s+s1}{\PYZsq{}}\PY{l+s+s1}{X1}\PY{l+s+s1}{\PYZsq{}}\PY{p}{:} \PY{p}{[}\PY{l+m+mi}{0}\PY{p}{,} \PY{l+m+mi}{2}\PY{p}{,} \PY{l+m+mi}{0}\PY{p}{,} \PY{l+m+mi}{0}\PY{p}{,} \PY{o}{\PYZhy{}}\PY{l+m+mi}{1}\PY{p}{,} \PY{l+m+mi}{1}\PY{p}{]}\PY{p}{,} 
                     \PY{l+s+s1}{\PYZsq{}}\PY{l+s+s1}{X2}\PY{l+s+s1}{\PYZsq{}}\PY{p}{:} \PY{p}{[}\PY{l+m+mi}{3}\PY{p}{,} \PY{l+m+mi}{0}\PY{p}{,} \PY{l+m+mi}{1}\PY{p}{,} \PY{l+m+mi}{1}\PY{p}{,} \PY{l+m+mi}{0}\PY{p}{,} \PY{l+m+mi}{1}\PY{p}{]}\PY{p}{,} 
                     \PY{l+s+s1}{\PYZsq{}}\PY{l+s+s1}{X3}\PY{l+s+s1}{\PYZsq{}}\PY{p}{:} \PY{p}{[}\PY{l+m+mi}{0}\PY{p}{,} \PY{l+m+mi}{0}\PY{p}{,} \PY{l+m+mi}{3}\PY{p}{,} \PY{l+m+mi}{2}\PY{p}{,} \PY{l+m+mi}{1}\PY{p}{,} \PY{l+m+mi}{1}\PY{p}{]}\PY{p}{\PYZcb{}}	

\PY{n}{table} \PY{o}{=} \PY{n}{pd}\PY{o}{.}\PY{n}{DataFrame}\PY{p}{(}\PY{n}{table\PYZus{}data}\PY{p}{,} \PY{n}{columns} \PY{o}{=} \PY{p}{[}\PY{l+s+s1}{\PYZsq{}}\PY{l+s+s1}{Y}\PY{l+s+s1}{\PYZsq{}}\PY{p}{,} \PY{l+s+s1}{\PYZsq{}}\PY{l+s+s1}{X1}\PY{l+s+s1}{\PYZsq{}}\PY{p}{,} \PY{l+s+s1}{\PYZsq{}}\PY{l+s+s1}{X2}\PY{l+s+s1}{\PYZsq{}}\PY{p}{,} \PY{l+s+s1}{\PYZsq{}}\PY{l+s+s1}{X3}\PY{l+s+s1}{\PYZsq{}}\PY{p}{]}\PY{p}{)}
\end{Verbatim}
\end{tcolorbox}

    \begin{tcolorbox}[breakable, size=fbox, boxrule=1pt, pad at break*=1mm,colback=cellbackground, colframe=cellborder]
\prompt{In}{incolor}{10}{\boxspacing}
\begin{Verbatim}[commandchars=\\\{\}]
\PY{c+c1}{\PYZsh{} (a) (2 pts) Compute the Euclidean distance between each observation and the test point, X1 = X2 = X3 = 0.}

\PY{c+c1}{\PYZsh{} Point X1 = X2 = X3 = 0}
\PY{n}{X1} \PY{o}{=} \PY{l+m+mi}{0}
\PY{n}{X2} \PY{o}{=} \PY{l+m+mi}{0}
\PY{n}{X3} \PY{o}{=} \PY{l+m+mi}{0}

\PY{c+c1}{\PYZsh{} Calculate the Euclidean distance between each observation and the test point}
\PY{n}{table}\PY{p}{[}\PY{l+s+s1}{\PYZsq{}}\PY{l+s+s1}{Euclidean Distance}\PY{l+s+s1}{\PYZsq{}}\PY{p}{]} \PY{o}{=} \PY{n}{np}\PY{o}{.}\PY{n}{sqrt}\PY{p}{(}\PY{p}{(}\PY{n}{table}\PY{p}{[}\PY{l+s+s1}{\PYZsq{}}\PY{l+s+s1}{X1}\PY{l+s+s1}{\PYZsq{}}\PY{p}{]} \PY{o}{\PYZhy{}} \PY{n}{X1}\PY{p}{)}\PY{o}{*}\PY{o}{*}\PY{l+m+mi}{2} \PY{o}{+} \PY{p}{(}\PY{n}{table}\PY{p}{[}\PY{l+s+s1}{\PYZsq{}}\PY{l+s+s1}{X2}\PY{l+s+s1}{\PYZsq{}}\PY{p}{]} \PY{o}{\PYZhy{}} \PY{n}{X2}\PY{p}{)}\PY{o}{*}\PY{o}{*}\PY{l+m+mi}{2} \PY{o}{+} \PY{p}{(}\PY{n}{table}\PY{p}{[}\PY{l+s+s1}{\PYZsq{}}\PY{l+s+s1}{X3}\PY{l+s+s1}{\PYZsq{}}\PY{p}{]} \PY{o}{\PYZhy{}} \PY{n}{X3}\PY{p}{)}\PY{o}{*}\PY{o}{*}\PY{l+m+mi}{2}\PY{p}{)}

\PY{c+c1}{\PYZsh{} Set the display format for float columns}
\PY{n}{pd}\PY{o}{.}\PY{n}{options}\PY{o}{.}\PY{n}{display}\PY{o}{.}\PY{n}{float\PYZus{}format} \PY{o}{=} \PY{l+s+s1}{\PYZsq{}}\PY{l+s+si}{\PYZob{}:.2f\PYZcb{}}\PY{l+s+s1}{\PYZsq{}}\PY{o}{.}\PY{n}{format}

\PY{c+c1}{\PYZsh{} Print the table}
\PY{n+nb}{print}\PY{p}{(}\PY{n}{table}\PY{p}{)}
\end{Verbatim}
\end{tcolorbox}

    \begin{Verbatim}[commandchars=\\\{\}]
       Y  X1  X2  X3  Euclidean Distance
0    Red   0   3   0                3.00
1    Red   2   0   0                2.00
2    Red   0   1   3                3.16
3  Green   0   1   2                2.24
4  Green  -1   0   1                1.41
5    Red   1   1   1                1.73
    \end{Verbatim}

    The calculated Euclidean distances from each point to the point of
interest are as follows:

\[
\begin{array}{cccc}
\hline
X_{1} & X_{2} & X_{3} & \text{Distance} & Y \\
\hline
0 & 3 & 0 & \sqrt{9} = 3 & \text{Red} \\
2 & 0 & 0 & \sqrt{4} = 2 & \text{Red} \\
0 & 1 & 3 & \sqrt{10} \approx 3.16 & \text{Red} \\
0 & 1 & 2 & \sqrt{5} \approx 2.24 & \text{Green} \\
-1 & 0 & 1 & \sqrt{2} \approx 1.41 & \text{Green} \\
1 & 1 & 1 & \sqrt{3} \approx 1.73 & \text{Red} \\
\hline
\end{array}
\]

    \begin{tcolorbox}[breakable, size=fbox, boxrule=1pt, pad at break*=1mm,colback=cellbackground, colframe=cellborder]
\prompt{In}{incolor}{11}{\boxspacing}
\begin{Verbatim}[commandchars=\\\{\}]
\PY{c+c1}{\PYZsh{} (b) (2 pts) What is our prediction with K = 1? Why?}

\PY{c+c1}{\PYZsh{} Sort the table by Euclidean Distance}
\PY{n}{table} \PY{o}{=} \PY{n}{table}\PY{o}{.}\PY{n}{sort\PYZus{}values}\PY{p}{(}\PY{n}{by}\PY{o}{=}\PY{p}{[}\PY{l+s+s1}{\PYZsq{}}\PY{l+s+s1}{Euclidean Distance}\PY{l+s+s1}{\PYZsq{}}\PY{p}{]}\PY{p}{)}

\PY{c+c1}{\PYZsh{} Print the table}
\PY{n+nb}{print}\PY{p}{(}\PY{n}{table}\PY{p}{)}

\PY{c+c1}{\PYZsh{} The prediction with K = 1 is Green because the closest observation is Green, with a distance of 1.41.}
\end{Verbatim}
\end{tcolorbox}

    \begin{Verbatim}[commandchars=\\\{\}]
       Y  X1  X2  X3  Euclidean Distance
4  Green  -1   0   1                1.41
5    Red   1   1   1                1.73
1    Red   2   0   0                2.00
3  Green   0   1   2                2.24
0    Red   0   3   0                3.00
2    Red   0   1   3                3.16
    \end{Verbatim}

    \begin{tcolorbox}[breakable, size=fbox, boxrule=1pt, pad at break*=1mm,colback=cellbackground, colframe=cellborder]
\prompt{In}{incolor}{12}{\boxspacing}
\begin{Verbatim}[commandchars=\\\{\}]
\PY{c+c1}{\PYZsh{} (c) (2 pts) What is our prediction with K = 3? Why?}

\PY{c+c1}{\PYZsh{} Print the table}
\PY{n+nb}{print}\PY{p}{(}\PY{n}{table}\PY{p}{)}

\PY{c+c1}{\PYZsh{} The prediction with K = 3 would be Red, from the 3 closest observations, 2 are Red and 1 is Green. The vote would be Red, hence the prediction. The values of these distances are 1.41, 1.73, and 2.00.}
\end{Verbatim}
\end{tcolorbox}

    \begin{Verbatim}[commandchars=\\\{\}]
       Y  X1  X2  X3  Euclidean Distance
4  Green  -1   0   1                1.41
5    Red   1   1   1                1.73
1    Red   2   0   0                2.00
3  Green   0   1   2                2.24
0    Red   0   3   0                3.00
2    Red   0   1   3                3.16
    \end{Verbatim}

    \begin{enumerate}
\def\labelenumi{\arabic{enumi}.}
\setcounter{enumi}{2}
\tightlist
\item
  (12 pts) When the number of features p is large, there tends to be a
  deterioration in the performance of KNN and other local approaches
  that perform prediction using only observations that are near the test
  observation for which a prediction must be made. This phenomenon is
  known as the curse of dimensionality, and it ties into the fact that
  non-parametric approaches often perform poorly when p is large.
\end{enumerate}

    \begin{enumerate}
\def\labelenumi{(\alph{enumi})}
\tightlist
\item
  (2pts) Suppose that we have a set of observations, each with
  measurements on p = 1 feature, X. We assume that X is uniformly
  (evenly) distributed on {[}0, 1{]}. Associated with each observation
  is a response value. Suppose that we wish to predict a test
  observation's response using only observations that are within 10\% of
  the range of X closest to that test observation. For instance, in
  order to predict the response for a test observation with X = 0.6, we
  will use observations in the range {[}0.55, 0.65{]}. On average, what
  fraction of the available observations will we use to make the
  prediction?
\end{enumerate}

    X is uniformly distributed on {[}0,1{]}.\\
We will predict the response for a test observation using the range
\([x - 0.05, x + 0.05]\)

Given X is uniformly distributed on {[}0,1{]}, the probability density
function (pdf) is:

\[
f(x) = \begin{cases}
1 & \text{if } 0 \leq x \leq 1 \\
0 & \text{otherwise}
\end{cases}
\]

Now, calculating the probability of the range the observation within the
10\% range:

\[  
\begin{align}
P(x - 0.05 \leq X \leq x + 0.05) &= \int_{x - 0.05}^{x + 0.05} f(x) dx \\
&= \int_{x - 0.05}^{x + 0.05} 1 dx \\
&= x + 0.05 - (x - 0.05) \\
&= 0.1
\end{align}
\]

So, the probability of the range the observation within the 10\% range
is 0.1.

    \begin{enumerate}
\def\labelenumi{(\alph{enumi})}
\setcounter{enumi}{1}
\tightlist
\item
  (2pts) Now suppose that we have a set of observations, each with
  measurements on p = 2 features, X1 and X2. We assume that predict a
  test observation's response using only observations that (X1,X2) are
  uniformly distributed on {[}0, 1{]} × {[}0, 1{]}. We wish to are
  within 10\% of the range of X1 and within 10\% of the range of X2
  closest to that test observation. For instance, in order to predict
  the response for a test observation with X1 = 0.6 and X2 = 0.35, we
  will use observations in the range {[}0.55, 0.65{]} for X1 and in the
  range {[}0.3, 0.4{]} for X2. On average, what fraction of the
  available observations will we use to make the prediction?
\end{enumerate}

    Both X1, X2 are uniformly distributed on {[}0,1{]}.\\
We will predict the response for a test observation using the range
\([x_{1} - 0.05, x_{1} + 0.05]\) for X1 and
\([x_{2} - 0.05, x_{2} + 0.05]\) for X2

Given X1, X2 are uniformly distributed on {[}0,1{]}, the probability
density function (pdf) is:

\[
f(x_{1}, x_{2}) = \begin{cases}
1 & \text{if } 0 \leq x_{1}, x_{2} \leq 1 \\
0 & \text{otherwise}
\end{cases}
\]

Now, calculating the probability of the range x1 {[}0.55, 0.65{]} and x2
{[}0.3, 0.4{]}:

\[
\begin{align}
P(0.6 - 0.05 \leq X_{1} \leq 0.6 + 0.05, 0.35 - 0.05 \leq X_{2} \leq 0.35 + 0.05) &= \int_{0.55}^{0.65} \int_{0.3}^{0.4} f(x_{1}, x_{2}) dx_{1} dx_{2} \\
&= \int_{0.55}^{0.65} \int_{0.3}^{0.4} 1 dx_{1} dx_{2} \\
&= \int_{0.55}^{0.65} x_{1} \Big|_{0.3}^{0.4} dx_{2} \\
&= \int_{0.55}^{0.65} 0.1 dx_{2} \\
&= 0.1 \int_{0.55}^{0.65} dx_{2} \\
&= 0.1 x_{2} \Big|_{0.55}^{0.65} \\
&= 0.1 (0.65 - 0.55) \\
&= 0.01
\end{align}
\]

We will use 1\% of the available observations to make the prediction
when considering both features, with a 10 percent range.

    \begin{enumerate}
\def\labelenumi{(\alph{enumi})}
\setcounter{enumi}{2}
\tightlist
\item
  (2pts) Now suppose that we have a set of observations on p = 100
  features. Again the observations are uniformly distributed on each
  feature, and again each feature ranges in value from 0 to 1. We wish
  to predict a test observation's response using observations within the
  10\% of each feature's range that is closest to that test observation.
  What fraction of the available observations will we use to make the
  prediction?
\end{enumerate}

    So, from the previous answer we know that:

For p=1, we use 10\% or 0.1 of the observations.\\
For p=2, we use 1\% or 0.01 of the observations.\\

These values are the product of the fractions of observations used in
each individual feature's dimension. This is because, given independence
and uniform distribution, the fraction of observations used in a
multi-dimensional space is the product of the probabilities in each
dimension.

Extending this to 100 dimensions, we get:

\[
\begin{align}
(0.1)^{p} &= 1.0 \times 10^{-100} \\
\end{align}
\]

So we will use \(1.0 \times 10^{-100}\) of the available observations to
make the prediction when considering 100 features, with a 10 percent
range.

    \begin{enumerate}
\def\labelenumi{(\alph{enumi})}
\setcounter{enumi}{3}
\tightlist
\item
  (3pts) Using your answers to parts (a)--(c), argue that a drawback of
  KNN when p is large is that there are very few training observations
  ``near'' any given test observation.
\end{enumerate}

    This is one of the challanges impose by the ``curse of dimensionality'',
as the number of features (or dimensions) increases, so does the
distance between the data points. KNN is approach that leverages on
spatial proximity to make predictions, so as the distance between the
data points increases, the accuracy of the predictions decreases. Even
on data sets with a large number of observations, if the number of
features is large, is unlikely that the data points will be close to
each other, because the distance is a exponential function of the number
of features.

    \begin{enumerate}
\def\labelenumi{(\alph{enumi})}
\setcounter{enumi}{4}
\tightlist
\item
  (3pts) Now suppose that we wish to make a prediction for a test
  observation by creating a p-dimensional hypercube centered around the
  test observation that contains, on average, 10\% of the training
  observations. For p =1, 2, and 100, what is the length of each side of
  the hypercube? Comment what happens to the length of the sides as
  \(\text{lim} p \rightarrow \infty\).
\end{enumerate}

    In order to calculate the lenth of the hypercube on each one of the p
values, let's first define the volume of the hypercube as:

\[
\begin{align}
V &= L^{p} \\
\end{align}
\]

Where L is the length of each side of the hypercube and p is the number
of dimensions.

We know that percentage of observations used in each individual
feature's dimension is 0.1, so:

\[
\begin{align}
L^{p} &= 0.1 \\
L &= 0.1^{\frac{1}{p}} \\
\end{align}
\]

Replacing the values of 1,2, and 100 for p, we get:

\[
\begin{align}
L_{p=1} &= 0.1^{\frac{1}{1}} = 0.1 \\
L_{p=2} &= 0.1^{\frac{1}{2}} = 0.316 \\
L_{p=100} &= 0.1^{\frac{1}{100}} = 0.977 \\
\end{align}
\]

This shows that as the value of features increases, so does the length
of the sides of the hypercube, getting closer to 1, representing the
entire range of the feature. This means that with more features, we
progresively need to consider a larger range of values to make
predictions, making the ``neighborhood'' of the KNN model larger, and
therefore less accurate.

    \begin{enumerate}
\def\labelenumi{\arabic{enumi}.}
\setcounter{enumi}{3}
\tightlist
\item
  (6 pts) Supoose you trained a classifier for a spam detection system.
  The prediction result on the test set is summarized in the following
  table.
\end{enumerate}

\[
\begin{array}{cccc}
\hline
\text{Predicted class} & \text{Spam} & \text{not Spam} \\
\hline
\text{Actual class Spam} & 8 & 2 \\
\text{Actual class not Spam} & 16 & 974 \\
\hline
\end{array}
\]

    Calculate

\begin{itemize}
\item
  \begin{enumerate}
  \def\labelenumi{(\alph{enumi})}
  \tightlist
  \item
    (2 pts) Accuracy
  \end{enumerate}
\item
  \begin{enumerate}
  \def\labelenumi{(\alph{enumi})}
  \setcounter{enumi}{1}
  \tightlist
  \item
    (2 pts) Precision
  \end{enumerate}
\item
  \begin{enumerate}
  \def\labelenumi{(\alph{enumi})}
  \setcounter{enumi}{2}
  \tightlist
  \item
    (2 pts) Recall
  \end{enumerate}
\end{itemize}

    \begin{enumerate}
\def\labelenumi{(\alph{enumi})}
\tightlist
\item
  Accuracy
\end{enumerate}

\[
\begin{align}
Accuracy &= \frac{TP + TN}{TP + TN + FP + FN} \\
&= \frac{8 + 974}{8 + 974 + 16 + 2} \\
&= \frac{982}{1000} \\
&= 0.982
\end{align}
\]

    \begin{enumerate}
\def\labelenumi{(\alph{enumi})}
\setcounter{enumi}{1}
\tightlist
\item
  Precision
\end{enumerate}

\[
\begin{align}
Precision &= \frac{TP}{TP + FP} \\
&= \frac{8}{8 + 16} \\
&= \frac{8}{24} \\
&= 0.333
\end{align}
\]

    \begin{enumerate}
\def\labelenumi{(\alph{enumi})}
\setcounter{enumi}{2}
\tightlist
\item
  Recall
\end{enumerate}

\[
\begin{align}
Recall &= \frac{TP}{TP + FN} \\
&= \frac{8}{8 + 2} \\
&= \frac{8}{10} \\
&= 0.8
\end{align}
\]

    \begin{enumerate}
\def\labelenumi{\arabic{enumi}.}
\setcounter{enumi}{4}
\tightlist
\item
  (9pts) Again, suppose you trained a classifier for a spam filter. The
  prediction result on the test set is summarized in the following
  table. Here, ''+'' represents spam, and ''-'' means not spam.
\end{enumerate}

\[
\begin{array}{cccc}
\hline
\text{Confidence Positive} & \text{Correct class} \\
\hline
0.95 & + \\
0.85 & + \\
0.8 & - \\
0.7 & + \\
0.55 & + \\
0.45 & - \\
0.4 & + \\
0.3 & + \\
0.2 & - \\
0.1 & - \\
\hline
\end{array}
\]

\begin{enumerate}
\def\labelenumi{(\alph{enumi})}
\tightlist
\item
  (6pts) Draw a ROC curve based on the above table.
\end{enumerate}

    \begin{tcolorbox}[breakable, size=fbox, boxrule=1pt, pad at break*=1mm,colback=cellbackground, colframe=cellborder]
\prompt{In}{incolor}{6}{\boxspacing}
\begin{Verbatim}[commandchars=\\\{\}]
\PY{c+c1}{\PYZsh{} Dataframe}

\PY{n}{table\PYZus{}data} \PY{o}{=} \PY{p}{\PYZob{}}\PY{l+s+s1}{\PYZsq{}}\PY{l+s+s1}{Confidence Positive}\PY{l+s+s1}{\PYZsq{}}\PY{p}{:} \PY{p}{[}\PY{l+m+mf}{0.95}\PY{p}{,} \PY{l+m+mf}{0.85}\PY{p}{,} \PY{l+m+mf}{0.8}\PY{p}{,} \PY{l+m+mf}{0.7}\PY{p}{,} \PY{l+m+mf}{0.55}\PY{p}{,} \PY{l+m+mf}{0.45}\PY{p}{,} \PY{l+m+mf}{0.4}\PY{p}{,} \PY{l+m+mf}{0.3}\PY{p}{,} \PY{l+m+mf}{0.2}\PY{p}{,} \PY{l+m+mf}{0.1}\PY{p}{]}\PY{p}{,}    
                \PY{l+s+s1}{\PYZsq{}}\PY{l+s+s1}{Correct Class}\PY{l+s+s1}{\PYZsq{}}\PY{p}{:} \PY{p}{[}\PY{l+s+s2}{\PYZdq{}}\PY{l+s+s2}{+}\PY{l+s+s2}{\PYZdq{}}\PY{p}{,} \PY{l+s+s2}{\PYZdq{}}\PY{l+s+s2}{+}\PY{l+s+s2}{\PYZdq{}}\PY{p}{,} \PY{l+s+s2}{\PYZdq{}}\PY{l+s+s2}{\PYZhy{}}\PY{l+s+s2}{\PYZdq{}}\PY{p}{,} \PY{l+s+s2}{\PYZdq{}}\PY{l+s+s2}{+}\PY{l+s+s2}{\PYZdq{}}\PY{p}{,} \PY{l+s+s2}{\PYZdq{}}\PY{l+s+s2}{+}\PY{l+s+s2}{\PYZdq{}}\PY{p}{,} \PY{l+s+s2}{\PYZdq{}}\PY{l+s+s2}{\PYZhy{}}\PY{l+s+s2}{\PYZdq{}}\PY{p}{,} \PY{l+s+s2}{\PYZdq{}}\PY{l+s+s2}{+}\PY{l+s+s2}{\PYZdq{}}\PY{p}{,} \PY{l+s+s2}{\PYZdq{}}\PY{l+s+s2}{+}\PY{l+s+s2}{\PYZdq{}}\PY{p}{,} \PY{l+s+s2}{\PYZdq{}}\PY{l+s+s2}{\PYZhy{}}\PY{l+s+s2}{\PYZdq{}}\PY{p}{,} \PY{l+s+s2}{\PYZdq{}}\PY{l+s+s2}{\PYZhy{}}\PY{l+s+s2}{\PYZdq{}}\PY{p}{]}\PY{p}{\PYZcb{}}	

\PY{n}{table} \PY{o}{=} \PY{n}{pd}\PY{o}{.}\PY{n}{DataFrame}\PY{p}{(}\PY{n}{table\PYZus{}data}\PY{p}{,} \PY{n}{columns} \PY{o}{=} \PY{p}{[}\PY{l+s+s1}{\PYZsq{}}\PY{l+s+s1}{Confidence Positive}\PY{l+s+s1}{\PYZsq{}}\PY{p}{,} \PY{l+s+s1}{\PYZsq{}}\PY{l+s+s1}{Correct Class}\PY{l+s+s1}{\PYZsq{}}\PY{p}{]}\PY{p}{)}
\end{Verbatim}
\end{tcolorbox}

    \begin{tcolorbox}[breakable, size=fbox, boxrule=1pt, pad at break*=1mm,colback=cellbackground, colframe=cellborder]
\prompt{In}{incolor}{7}{\boxspacing}
\begin{Verbatim}[commandchars=\\\{\}]
\PY{c+c1}{\PYZsh{} Initialize variables}
\PY{n}{tpr\PYZus{}list} \PY{o}{=} \PY{p}{[}\PY{p}{]}
\PY{n}{fpr\PYZus{}list} \PY{o}{=} \PY{p}{[}\PY{p}{]}

\PY{n}{positive} \PY{o}{=} \PY{n+nb}{sum}\PY{p}{(}\PY{n}{table}\PY{p}{[}\PY{l+s+s1}{\PYZsq{}}\PY{l+s+s1}{Correct Class}\PY{l+s+s1}{\PYZsq{}}\PY{p}{]} \PY{o}{==} \PY{l+s+s1}{\PYZsq{}}\PY{l+s+s1}{+}\PY{l+s+s1}{\PYZsq{}}\PY{p}{)}
\PY{n}{negative} \PY{o}{=} \PY{n+nb}{sum}\PY{p}{(}\PY{n}{table}\PY{p}{[}\PY{l+s+s1}{\PYZsq{}}\PY{l+s+s1}{Correct Class}\PY{l+s+s1}{\PYZsq{}}\PY{p}{]} \PY{o}{==} \PY{l+s+s1}{\PYZsq{}}\PY{l+s+s1}{\PYZhy{}}\PY{l+s+s1}{\PYZsq{}}\PY{p}{)}

\PY{c+c1}{\PYZsh{} Iterate through thresholds}
\PY{k}{for} \PY{n}{threshold} \PY{o+ow}{in} \PY{n}{table}\PY{p}{[}\PY{l+s+s1}{\PYZsq{}}\PY{l+s+s1}{Confidence Positive}\PY{l+s+s1}{\PYZsq{}}\PY{p}{]}\PY{p}{:}
    \PY{n}{pred\PYZus{}class} \PY{o}{=} \PY{p}{[} \PY{l+s+s1}{\PYZsq{}}\PY{l+s+s1}{+}\PY{l+s+s1}{\PYZsq{}} \PY{k}{if} \PY{n}{x} \PY{o}{\PYZgt{}}\PY{o}{=} \PY{n}{threshold} \PY{k}{else} \PY{l+s+s1}{\PYZsq{}}\PY{l+s+s1}{\PYZhy{}}\PY{l+s+s1}{\PYZsq{}} \PY{k}{for} \PY{n}{x} \PY{o+ow}{in} \PY{n}{table}\PY{p}{[}\PY{l+s+s1}{\PYZsq{}}\PY{l+s+s1}{Confidence Positive}\PY{l+s+s1}{\PYZsq{}}\PY{p}{]}\PY{p}{]}
    \PY{n}{table}\PY{p}{[}\PY{l+s+s1}{\PYZsq{}}\PY{l+s+s1}{Predicted Class}\PY{l+s+s1}{\PYZsq{}}\PY{p}{]} \PY{o}{=} \PY{n}{pred\PYZus{}class}
    
    \PY{c+c1}{\PYZsh{} Compute True Positive, False Positive}
    \PY{n}{tp} \PY{o}{=} \PY{n+nb}{sum}\PY{p}{(}\PY{p}{(}\PY{n}{table}\PY{p}{[}\PY{l+s+s1}{\PYZsq{}}\PY{l+s+s1}{Correct Class}\PY{l+s+s1}{\PYZsq{}}\PY{p}{]} \PY{o}{==} \PY{l+s+s1}{\PYZsq{}}\PY{l+s+s1}{+}\PY{l+s+s1}{\PYZsq{}}\PY{p}{)} \PY{o}{\PYZam{}} \PY{p}{(}\PY{n}{table}\PY{p}{[}\PY{l+s+s1}{\PYZsq{}}\PY{l+s+s1}{Predicted Class}\PY{l+s+s1}{\PYZsq{}}\PY{p}{]} \PY{o}{==} \PY{l+s+s1}{\PYZsq{}}\PY{l+s+s1}{+}\PY{l+s+s1}{\PYZsq{}}\PY{p}{)}\PY{p}{)}
    \PY{n}{fp} \PY{o}{=} \PY{n+nb}{sum}\PY{p}{(}\PY{p}{(}\PY{n}{table}\PY{p}{[}\PY{l+s+s1}{\PYZsq{}}\PY{l+s+s1}{Correct Class}\PY{l+s+s1}{\PYZsq{}}\PY{p}{]} \PY{o}{==} \PY{l+s+s1}{\PYZsq{}}\PY{l+s+s1}{\PYZhy{}}\PY{l+s+s1}{\PYZsq{}}\PY{p}{)} \PY{o}{\PYZam{}} \PY{p}{(}\PY{n}{table}\PY{p}{[}\PY{l+s+s1}{\PYZsq{}}\PY{l+s+s1}{Predicted Class}\PY{l+s+s1}{\PYZsq{}}\PY{p}{]} \PY{o}{==} \PY{l+s+s1}{\PYZsq{}}\PY{l+s+s1}{+}\PY{l+s+s1}{\PYZsq{}}\PY{p}{)}\PY{p}{)}
    
    \PY{c+c1}{\PYZsh{} Compute True Positive Rate, False Positive Rate}
    \PY{n}{tpr} \PY{o}{=} \PY{n}{tp} \PY{o}{/} \PY{n}{positive}
    \PY{n}{fpr} \PY{o}{=} \PY{n}{fp} \PY{o}{/} \PY{n}{negative}
    
    \PY{n}{tpr\PYZus{}list}\PY{o}{.}\PY{n}{append}\PY{p}{(}\PY{n}{tpr}\PY{p}{)}
    \PY{n}{fpr\PYZus{}list}\PY{o}{.}\PY{n}{append}\PY{p}{(}\PY{n}{fpr}\PY{p}{)}

\PY{c+c1}{\PYZsh{} Add (0, 0) and (1, 1) to make the plot start at the origin and end at top right}
\PY{n}{tpr\PYZus{}list} \PY{o}{=} \PY{p}{[}\PY{l+m+mi}{0}\PY{p}{]} \PY{o}{+} \PY{n}{tpr\PYZus{}list} \PY{o}{+} \PY{p}{[}\PY{l+m+mi}{1}\PY{p}{]}
\PY{n}{fpr\PYZus{}list} \PY{o}{=} \PY{p}{[}\PY{l+m+mi}{0}\PY{p}{]} \PY{o}{+} \PY{n}{fpr\PYZus{}list} \PY{o}{+} \PY{p}{[}\PY{l+m+mi}{1}\PY{p}{]}

\PY{c+c1}{\PYZsh{} Plotting}
\PY{n}{plt}\PY{o}{.}\PY{n}{figure}\PY{p}{(}\PY{n}{figsize}\PY{o}{=}\PY{p}{(}\PY{l+m+mi}{8}\PY{p}{,} \PY{l+m+mi}{8}\PY{p}{)}\PY{p}{)}
\PY{n}{plt}\PY{o}{.}\PY{n}{plot}\PY{p}{(}\PY{n}{fpr\PYZus{}list}\PY{p}{,} \PY{n}{tpr\PYZus{}list}\PY{p}{,} \PY{n}{marker}\PY{o}{=}\PY{l+s+s1}{\PYZsq{}}\PY{l+s+s1}{o}\PY{l+s+s1}{\PYZsq{}}\PY{p}{)}
\PY{n}{plt}\PY{o}{.}\PY{n}{plot}\PY{p}{(}\PY{p}{[}\PY{l+m+mi}{0}\PY{p}{,} \PY{l+m+mi}{1}\PY{p}{]}\PY{p}{,} \PY{p}{[}\PY{l+m+mi}{0}\PY{p}{,} \PY{l+m+mi}{1}\PY{p}{]}\PY{p}{,} \PY{n}{linestyle}\PY{o}{=}\PY{l+s+s1}{\PYZsq{}}\PY{l+s+s1}{\PYZhy{}\PYZhy{}}\PY{l+s+s1}{\PYZsq{}}\PY{p}{,} \PY{n}{color}\PY{o}{=}\PY{l+s+s1}{\PYZsq{}}\PY{l+s+s1}{gray}\PY{l+s+s1}{\PYZsq{}}\PY{p}{)}
\PY{n}{plt}\PY{o}{.}\PY{n}{title}\PY{p}{(}\PY{l+s+s1}{\PYZsq{}}\PY{l+s+s1}{ROC Curve}\PY{l+s+s1}{\PYZsq{}}\PY{p}{)}
\PY{n}{plt}\PY{o}{.}\PY{n}{xlabel}\PY{p}{(}\PY{l+s+s1}{\PYZsq{}}\PY{l+s+s1}{False Positive Rate}\PY{l+s+s1}{\PYZsq{}}\PY{p}{)}
\PY{n}{plt}\PY{o}{.}\PY{n}{ylabel}\PY{p}{(}\PY{l+s+s1}{\PYZsq{}}\PY{l+s+s1}{True Positive Rate}\PY{l+s+s1}{\PYZsq{}}\PY{p}{)}
\PY{n}{plt}\PY{o}{.}\PY{n}{xlim}\PY{p}{(}\PY{p}{[}\PY{l+m+mi}{0}\PY{p}{,} \PY{l+m+mi}{1}\PY{p}{]}\PY{p}{)}
\PY{n}{plt}\PY{o}{.}\PY{n}{ylim}\PY{p}{(}\PY{p}{[}\PY{l+m+mi}{0}\PY{p}{,} \PY{l+m+mi}{1}\PY{p}{]}\PY{p}{)}
\PY{n}{plt}\PY{o}{.}\PY{n}{grid}\PY{p}{(}\PY{k+kc}{True}\PY{p}{)}
\PY{n}{plt}\PY{o}{.}\PY{n}{show}\PY{p}{(}\PY{p}{)}
\end{Verbatim}
\end{tcolorbox}

    \begin{center}
    \adjustimage{max size={0.9\linewidth}{0.9\paperheight}}{Homework 3 - Dario Placencio_files/Homework 3 - Dario Placencio_27_0.png}
    \end{center}
    { \hspace*{\fill} \\}
    
    \begin{enumerate}
\def\labelenumi{(\alph{enumi})}
\setcounter{enumi}{1}
\tightlist
\item
  (3pts) (Real-world open question) Suppose you want to choose a
  threshold parameter so that mails with confidence positives above the
  threshold can be classified as spam. Which value will you choose?
  Justify your answer based on the ROC curve.
\end{enumerate}

    Considering the context, what I would like to avoid is to have False
Positives, emails that are actually not Spam, and that I would be
missing from my inbox because of missclassification. So, I would choose
a threshold that would minimize the False Positive Rate (FPR), based on
the values of the ROC curve, this treshold would be .85, which would
give me a FPR of 0, but in comparison a little bit of a better
classification of the True Positives (TPR) than the threshold of 0.95.

    \begin{enumerate}
\def\labelenumi{\arabic{enumi}.}
\setcounter{enumi}{5}
\tightlist
\item
  (8 pts) In this problem, we will walk through a single step of the
  gradient descent algorithm for logistic regression. As a reminder,
\end{enumerate}

\[\hat{y} = f(x, \theta)\] \[f(x;\theta) = \sigma(\theta^\top x)\]
\[\text{Cross entropy loss } L(\hat{y}, y) = -[y \log  \hat{y} + (1-y)\log(1-\hat{y})]\]
\[\text{The single update step } \theta^{t+1} = \theta^{t} - \eta \nabla_{\theta} L(f(x;\theta), y) \]

    \begin{enumerate}
\def\labelenumi{(\alph{enumi})}
\tightlist
\item
  Compute the first gradient \(\nabla_{\theta} L(f(x;\theta), y)\).
\end{enumerate}

    The first gradient of the loss function is:

\[
\begin{align}
\nabla_{\theta} L(f(x;\theta), y) &= \frac{\partial}{\partial \theta} L(f(x;\theta), y) \\
&= \frac{\partial}{\partial \theta} -[y \log  \hat{y} + (1-y)\log(1-\hat{y})] \\
&= -[y \frac{\partial}{\partial \theta} \log  \hat{y} + (1-y)\frac{\partial}{\partial \theta} \log(1-\hat{y})] \\
&= -[y \frac{\partial}{\partial \theta} \log  \sigma(\theta^\top x) + (1-y)\frac{\partial}{\partial \theta} \log(1-\sigma(\theta^\top x))] \\
&= -[y \frac{1}{\sigma(\theta^\top x)} \frac{\partial}{\partial \theta} \sigma(\theta^\top x) + (1-y)\frac{1}{1-\sigma(\theta^\top x)}\frac{\partial}{\partial \theta} \log(1-\sigma(\theta^\top x))] \\
&= -[y \frac{1}{\sigma(\theta^\top x)} \sigma(\theta^\top x) (1 - \sigma(\theta^\top x)) x + (1-y)\frac{1}{1-\sigma(\theta^\top x)}(-\sigma(\theta^\top x) (1 - \sigma(\theta^\top x)) x)] \\
&= -[y (1 - \sigma(\theta^\top x)) x + (1-y)(-\sigma(\theta^\top x) x)] \\
&= -[y x - y \sigma(\theta^\top x) x - \sigma(\theta^\top x) x + y \sigma(\theta^\top x) x] \\
&= -[y x - \sigma(\theta^\top x) x] \\
&= -[y - \sigma(\theta^\top x)] x \\
&= -[y - f(x;\theta)] x \\
\end{align}
\]

    \begin{enumerate}
\def\labelenumi{(\alph{enumi})}
\setcounter{enumi}{1}
\tightlist
\item
  Now assume a two dimensional input. After including a bias parameter
  for the first dimension, we will have
\end{enumerate}

\(\theta\in\mathbb{R}^3\).
\[ \text{Initial parameters : }  \theta^{0}=[0, 0, 0]\]
\[ \text{Learning rate }\eta=0.1\]
\[ \text{data example : } x=[1, 3, 2], y=1\]

Compute the updated parameter vector \(\theta^{1}\) from the single
update step.

    We know that:

\[
\begin{align}
\theta^{t+1} &= \theta^{t} - \eta \nabla_{\theta} L(f(x;\theta), y) \\
\end{align}
\]

So, we need to first calculate \(\hat{y}\) - \(f(x;\theta)\). We know
that:

\[
\begin{align}
z &= \theta^\top x \\
\end{align}
\]

So, for the given data example, we have:

\[
\begin{align}
z &= \theta^\top x \\
&= [0, 0, 0]^\top [1, 3, 2] \\
&= 0 \\
\end{align}
\]

And, we know that:

\[
\begin{align}
\hat{y} &= \sigma(z) \\
&= \sigma(0) \\
&= frac{1}{1 + e^{-0}} \\
&= 0.5 \\
\end{align}
\]

Now, we can calculate the updated parameter vector \(\theta^{1}\):

\[
\begin{align}
\theta^{1} &= \theta^{0} - \eta \nabla_{\theta} L(f(x;\theta), y) \\
&= [0, 0, 0] - 0.1 (-[y - f(x;\theta)] x) \\
&= [0, 0, 0] - 0.1 (-[1 - 0.5] [1, 3, 2]) \\
&= [0, 0, 0] - 0.1 (-[0.5] [1, 3, 2]) \\
&= [0, 0, 0] - 0.1 (-[0.5, 1.5, 1]) \\
&= [0, 0, 0] - [-0.05, -0.15, -0.1] \\
&= [0.05, 0.15, 0.1] \\
\end{align}
\]

    \hypertarget{programing-part}{%
\subsection{2. Programing Part}\label{programing-part}}

    \begin{enumerate}
\def\labelenumi{\arabic{enumi}.}
\tightlist
\item
  (10 pts) Use the whole D2z.txt as training set. Use Euclidean distance
  (i.e. \(A=I\)). Visualize the predictions of 1NN on a 2D grid
  \([-2:0.1:2]^2\). That is, you should produce test points whose first
  feature goes over \(-2, -1.9, -1.8, \ldots, 1.9, 2\), so does the
  second feature independent of the first feature. You should overlay
  the training set in the plot, just make sure we can tell which points
  are training, which are grid.
\end{enumerate}

    \begin{tcolorbox}[breakable, size=fbox, boxrule=1pt, pad at break*=1mm,colback=cellbackground, colframe=cellborder]
\prompt{In}{incolor}{11}{\boxspacing}
\begin{Verbatim}[commandchars=\\\{\}]
\PY{c+c1}{\PYZsh{} Read D2z txt}
\PY{n}{D2z} \PY{o}{=} \PY{n}{pd}\PY{o}{.}\PY{n}{read\PYZus{}csv}\PY{p}{(}\PY{l+s+s1}{\PYZsq{}}\PY{l+s+s1}{D2z.txt}\PY{l+s+s1}{\PYZsq{}}\PY{p}{,} \PY{n}{sep}\PY{o}{=}\PY{l+s+s1}{\PYZsq{}}\PY{l+s+s1}{ }\PY{l+s+s1}{\PYZsq{}}\PY{p}{,} \PY{n}{header}\PY{o}{=}\PY{k+kc}{None}\PY{p}{)}
\end{Verbatim}
\end{tcolorbox}

    \begin{tcolorbox}[breakable, size=fbox, boxrule=1pt, pad at break*=1mm,colback=cellbackground, colframe=cellborder]
\prompt{In}{incolor}{12}{\boxspacing}
\begin{Verbatim}[commandchars=\\\{\}]
\PY{c+c1}{\PYZsh{} Add a column names}
\PY{n}{D2z}\PY{o}{.}\PY{n}{columns} \PY{o}{=} \PY{p}{[}\PY{l+s+s1}{\PYZsq{}}\PY{l+s+s1}{X1}\PY{l+s+s1}{\PYZsq{}}\PY{p}{,} \PY{l+s+s1}{\PYZsq{}}\PY{l+s+s1}{X2}\PY{l+s+s1}{\PYZsq{}}\PY{p}{,} \PY{l+s+s1}{\PYZsq{}}\PY{l+s+s1}{Label}\PY{l+s+s1}{\PYZsq{}}\PY{p}{]}
\end{Verbatim}
\end{tcolorbox}

    \begin{tcolorbox}[breakable, size=fbox, boxrule=1pt, pad at break*=1mm,colback=cellbackground, colframe=cellborder]
\prompt{In}{incolor}{13}{\boxspacing}
\begin{Verbatim}[commandchars=\\\{\}]
\PY{c+c1}{\PYZsh{} Print the first 5 rows}
\PY{n+nb}{print}\PY{p}{(}\PY{n}{D2z}\PY{o}{.}\PY{n}{head}\PY{p}{(}\PY{p}{)}\PY{p}{)}
\end{Verbatim}
\end{tcolorbox}

    \begin{Verbatim}[commandchars=\\\{\}]
         X1        X2  Label
0 -0.333338 -0.087171      0
1 -1.531730  0.358194      1
2  1.549590 -0.364050      0
3 -1.349910  0.252063      1
4  1.283850 -0.531146      0
    \end{Verbatim}

    \begin{tcolorbox}[breakable, size=fbox, boxrule=1pt, pad at break*=1mm,colback=cellbackground, colframe=cellborder]
\prompt{In}{incolor}{14}{\boxspacing}
\begin{Verbatim}[commandchars=\\\{\}]
\PY{c+c1}{\PYZsh{} Create a grid}
\PY{n}{x}\PY{p}{,} \PY{n}{y} \PY{o}{=} \PY{n}{np}\PY{o}{.}\PY{n}{meshgrid}\PY{p}{(}\PY{n}{np}\PY{o}{.}\PY{n}{arange}\PY{p}{(}\PY{o}{\PYZhy{}}\PY{l+m+mi}{2}\PY{p}{,} \PY{l+m+mf}{2.1}\PY{p}{,} \PY{l+m+mf}{0.1}\PY{p}{)}\PY{p}{,} \PY{n}{np}\PY{o}{.}\PY{n}{arange}\PY{p}{(}\PY{o}{\PYZhy{}}\PY{l+m+mi}{2}\PY{p}{,} \PY{l+m+mf}{2.1}\PY{p}{,} \PY{l+m+mf}{0.1}\PY{p}{)}\PY{p}{)}
\PY{n}{grid} \PY{o}{=} \PY{n}{np}\PY{o}{.}\PY{n}{c\PYZus{}}\PY{p}{[}\PY{n}{x}\PY{o}{.}\PY{n}{ravel}\PY{p}{(}\PY{p}{)}\PY{p}{,} \PY{n}{y}\PY{o}{.}\PY{n}{ravel}\PY{p}{(}\PY{p}{)}\PY{p}{]}
\end{Verbatim}
\end{tcolorbox}

    \begin{tcolorbox}[breakable, size=fbox, boxrule=1pt, pad at break*=1mm,colback=cellbackground, colframe=cellborder]
\prompt{In}{incolor}{15}{\boxspacing}
\begin{Verbatim}[commandchars=\\\{\}]
\PY{c+c1}{\PYZsh{} Prediction List}
\PY{n}{predictions} \PY{o}{=} \PY{p}{[}\PY{p}{]}
\end{Verbatim}
\end{tcolorbox}

    \begin{tcolorbox}[breakable, size=fbox, boxrule=1pt, pad at break*=1mm,colback=cellbackground, colframe=cellborder]
\prompt{In}{incolor}{17}{\boxspacing}
\begin{Verbatim}[commandchars=\\\{\}]
\PY{c+c1}{\PYZsh{} 1NN and Predictions}
\PY{k}{for} \PY{n}{point} \PY{o+ow}{in} \PY{n}{grid}\PY{p}{:}
    \PY{c+c1}{\PYZsh{} Calculate the Euclidean distance between each observation and the test point}
    \PY{n}{D2z}\PY{p}{[}\PY{l+s+s1}{\PYZsq{}}\PY{l+s+s1}{Euclidean Distance}\PY{l+s+s1}{\PYZsq{}}\PY{p}{]} \PY{o}{=} \PY{n}{np}\PY{o}{.}\PY{n}{sqrt}\PY{p}{(}\PY{p}{(}\PY{n}{D2z}\PY{p}{[}\PY{l+s+s1}{\PYZsq{}}\PY{l+s+s1}{X1}\PY{l+s+s1}{\PYZsq{}}\PY{p}{]} \PY{o}{\PYZhy{}} \PY{n}{point}\PY{p}{[}\PY{l+m+mi}{0}\PY{p}{]}\PY{p}{)}\PY{o}{*}\PY{o}{*}\PY{l+m+mi}{2} \PY{o}{+} \PY{p}{(}\PY{n}{D2z}\PY{p}{[}\PY{l+s+s1}{\PYZsq{}}\PY{l+s+s1}{X2}\PY{l+s+s1}{\PYZsq{}}\PY{p}{]} \PY{o}{\PYZhy{}} \PY{n}{point}\PY{p}{[}\PY{l+m+mi}{1}\PY{p}{]}\PY{p}{)}\PY{o}{*}\PY{o}{*}\PY{l+m+mi}{2}\PY{p}{)}
    
    \PY{c+c1}{\PYZsh{} Sort the table by Euclidean Distance}
    \PY{n}{D2z} \PY{o}{=} \PY{n}{D2z}\PY{o}{.}\PY{n}{sort\PYZus{}values}\PY{p}{(}\PY{n}{by}\PY{o}{=}\PY{p}{[}\PY{l+s+s1}{\PYZsq{}}\PY{l+s+s1}{Euclidean Distance}\PY{l+s+s1}{\PYZsq{}}\PY{p}{]}\PY{p}{)}
    
    \PY{c+c1}{\PYZsh{} Append the prediction to the list}
    \PY{n}{predictions}\PY{o}{.}\PY{n}{append}\PY{p}{(}\PY{n}{D2z}\PY{o}{.}\PY{n}{iloc}\PY{p}{[}\PY{l+m+mi}{0}\PY{p}{]}\PY{p}{[}\PY{l+s+s1}{\PYZsq{}}\PY{l+s+s1}{Label}\PY{l+s+s1}{\PYZsq{}}\PY{p}{]}\PY{p}{)}

\PY{c+c1}{\PYZsh{} Reshape the predictions}
\PY{n}{predictions} \PY{o}{=} \PY{n}{np}\PY{o}{.}\PY{n}{array}\PY{p}{(}\PY{n}{predictions}\PY{p}{)}\PY{o}{.}\PY{n}{reshape}\PY{p}{(}\PY{n}{x}\PY{o}{.}\PY{n}{shape}\PY{p}{)}
\end{Verbatim}
\end{tcolorbox}

    \begin{tcolorbox}[breakable, size=fbox, boxrule=1pt, pad at break*=1mm,colback=cellbackground, colframe=cellborder]
\prompt{In}{incolor}{27}{\boxspacing}
\begin{Verbatim}[commandchars=\\\{\}]
\PY{c+c1}{\PYZsh{} Plotting}
\PY{n}{plt}\PY{o}{.}\PY{n}{figure}\PY{p}{(}\PY{n}{figsize}\PY{o}{=}\PY{p}{(}\PY{l+m+mi}{8}\PY{p}{,} \PY{l+m+mi}{8}\PY{p}{)}\PY{p}{)}
\PY{n}{plt}\PY{o}{.}\PY{n}{contourf}\PY{p}{(}\PY{n}{x}\PY{p}{,} \PY{n}{y}\PY{p}{,} \PY{n}{predictions}\PY{p}{,} \PY{n}{cmap}\PY{o}{=}\PY{l+s+s1}{\PYZsq{}}\PY{l+s+s1}{coolwarm}\PY{l+s+s1}{\PYZsq{}}\PY{p}{,} \PY{n}{alpha}\PY{o}{=}\PY{l+m+mf}{0.5}\PY{p}{)}

\PY{c+c1}{\PYZsh{} Separate data by label for legend}
\PY{n}{class0} \PY{o}{=} \PY{n}{D2z}\PY{p}{[}\PY{n}{D2z}\PY{p}{[}\PY{l+s+s1}{\PYZsq{}}\PY{l+s+s1}{Label}\PY{l+s+s1}{\PYZsq{}}\PY{p}{]} \PY{o}{==} \PY{l+m+mi}{0}\PY{p}{]}
\PY{n}{class1} \PY{o}{=} \PY{n}{D2z}\PY{p}{[}\PY{n}{D2z}\PY{p}{[}\PY{l+s+s1}{\PYZsq{}}\PY{l+s+s1}{Label}\PY{l+s+s1}{\PYZsq{}}\PY{p}{]} \PY{o}{==} \PY{l+m+mi}{1}\PY{p}{]}

\PY{c+c1}{\PYZsh{} Create separate scatter plots for each class and label them}
\PY{n}{plt}\PY{o}{.}\PY{n}{scatter}\PY{p}{(}\PY{n}{class0}\PY{p}{[}\PY{l+s+s1}{\PYZsq{}}\PY{l+s+s1}{X1}\PY{l+s+s1}{\PYZsq{}}\PY{p}{]}\PY{p}{,} \PY{n}{class0}\PY{p}{[}\PY{l+s+s1}{\PYZsq{}}\PY{l+s+s1}{X2}\PY{l+s+s1}{\PYZsq{}}\PY{p}{]}\PY{p}{,} \PY{n}{color}\PY{o}{=}\PY{l+s+s1}{\PYZsq{}}\PY{l+s+s1}{red}\PY{l+s+s1}{\PYZsq{}}\PY{p}{,} \PY{n}{label}\PY{o}{=}\PY{l+s+s1}{\PYZsq{}}\PY{l+s+s1}{Class 0}\PY{l+s+s1}{\PYZsq{}}\PY{p}{)}
\PY{n}{plt}\PY{o}{.}\PY{n}{scatter}\PY{p}{(}\PY{n}{class1}\PY{p}{[}\PY{l+s+s1}{\PYZsq{}}\PY{l+s+s1}{X1}\PY{l+s+s1}{\PYZsq{}}\PY{p}{]}\PY{p}{,} \PY{n}{class1}\PY{p}{[}\PY{l+s+s1}{\PYZsq{}}\PY{l+s+s1}{X2}\PY{l+s+s1}{\PYZsq{}}\PY{p}{]}\PY{p}{,} \PY{n}{color}\PY{o}{=}\PY{l+s+s1}{\PYZsq{}}\PY{l+s+s1}{blue}\PY{l+s+s1}{\PYZsq{}}\PY{p}{,} \PY{n}{label}\PY{o}{=}\PY{l+s+s1}{\PYZsq{}}\PY{l+s+s1}{Class 1}\PY{l+s+s1}{\PYZsq{}}\PY{p}{)}

\PY{c+c1}{\PYZsh{} Test point}
\PY{n}{plt}\PY{o}{.}\PY{n}{scatter}\PY{p}{(}\PY{l+m+mi}{0}\PY{p}{,} \PY{l+m+mi}{0}\PY{p}{,} \PY{n}{c}\PY{o}{=}\PY{l+s+s1}{\PYZsq{}}\PY{l+s+s1}{black}\PY{l+s+s1}{\PYZsq{}}\PY{p}{,} \PY{n}{marker}\PY{o}{=}\PY{l+s+s1}{\PYZsq{}}\PY{l+s+s1}{x}\PY{l+s+s1}{\PYZsq{}}\PY{p}{,} \PY{n}{s}\PY{o}{=}\PY{l+m+mi}{100}\PY{p}{,} \PY{n}{label}\PY{o}{=}\PY{l+s+s1}{\PYZsq{}}\PY{l+s+s1}{Test Point}\PY{l+s+s1}{\PYZsq{}}\PY{p}{)}

\PY{n}{plt}\PY{o}{.}\PY{n}{xlabel}\PY{p}{(}\PY{l+s+s1}{\PYZsq{}}\PY{l+s+s1}{X1}\PY{l+s+s1}{\PYZsq{}}\PY{p}{)}
\PY{n}{plt}\PY{o}{.}\PY{n}{ylabel}\PY{p}{(}\PY{l+s+s1}{\PYZsq{}}\PY{l+s+s1}{X2}\PY{l+s+s1}{\PYZsq{}}\PY{p}{)}   
\PY{n}{plt}\PY{o}{.}\PY{n}{xlim}\PY{p}{(}\PY{p}{[}\PY{o}{\PYZhy{}}\PY{l+m+mi}{2}\PY{p}{,} \PY{l+m+mi}{2}\PY{p}{]}\PY{p}{)}
\PY{n}{plt}\PY{o}{.}\PY{n}{ylim}\PY{p}{(}\PY{p}{[}\PY{o}{\PYZhy{}}\PY{l+m+mi}{2}\PY{p}{,} \PY{l+m+mi}{2}\PY{p}{]}\PY{p}{)}
\PY{n}{plt}\PY{o}{.}\PY{n}{grid}\PY{p}{(}\PY{k+kc}{True}\PY{p}{)}
\PY{n}{plt}\PY{o}{.}\PY{n}{legend}\PY{p}{(}\PY{p}{)} 
\PY{n}{plt}\PY{o}{.}\PY{n}{show}\PY{p}{(}\PY{p}{)}\PY{p}{;}
\end{Verbatim}
\end{tcolorbox}

    \begin{center}
    \adjustimage{max size={0.9\linewidth}{0.9\paperheight}}{Homework 3 - Dario Placencio_files/Homework 3 - Dario Placencio_43_0.png}
    \end{center}
    { \hspace*{\fill} \\}
    
    \{Spam filter\}

Now, we will use `emails.csv' as our dataset. The description is as
follows.

\begin{itemize}
\tightlist
\item
  Task: spam detection
\item
  The number of rows: 5000
\item
  The number of features: 3000 (Word frequency in each email)
\item
  The label (y) column name: `Predictor'
\item
  For a single training/test set split, use Email 1-4000 as the training
  set, Email 4001-5000 as the test set.
\item
  For 5-fold cross validation, split dataset in the following way.

  \begin{itemize}
  \tightlist
  \item
    Fold 1, test set: Email 1-1000, training set: the rest (Email
    1001-5000)
  \item
    Fold 2, test set: Email 1000-2000, training set: the rest
  \item
    Fold 3, test set: Email 2000-3000, training set: the rest
  \item
    Fold 4, test set: Email 3000-4000, training set: the rest
  \item
    Fold 5, test set: Email 4000-5000, training set: the rest
  \end{itemize}
\end{itemize}

    \begin{tcolorbox}[breakable, size=fbox, boxrule=1pt, pad at break*=1mm,colback=cellbackground, colframe=cellborder]
\prompt{In}{incolor}{3}{\boxspacing}
\begin{Verbatim}[commandchars=\\\{\}]
\PY{c+c1}{\PYZsh{} Read emails.csv}
\PY{n}{emails} \PY{o}{=} \PY{n}{pd}\PY{o}{.}\PY{n}{read\PYZus{}csv}\PY{p}{(}\PY{l+s+s1}{\PYZsq{}}\PY{l+s+s1}{emails.csv}\PY{l+s+s1}{\PYZsq{}}\PY{p}{)}
\end{Verbatim}
\end{tcolorbox}

    \begin{tcolorbox}[breakable, size=fbox, boxrule=1pt, pad at break*=1mm,colback=cellbackground, colframe=cellborder]
\prompt{In}{incolor}{4}{\boxspacing}
\begin{Verbatim}[commandchars=\\\{\}]
\PY{c+c1}{\PYZsh{} Print the first 5 rows}
\PY{n+nb}{print}\PY{p}{(}\PY{n}{emails}\PY{o}{.}\PY{n}{head}\PY{p}{(}\PY{p}{)}\PY{p}{)}
\end{Verbatim}
\end{tcolorbox}

    \begin{Verbatim}[commandchars=\\\{\}]
  Email No.  the  to  ect  and  for  of    a  you  hou  {\ldots}  connevey  jay  \textbackslash{}
0   Email 1    0   0    1    0    0   0    2    0    0  {\ldots}         0    0
1   Email 2    8  13   24    6    6   2  102    1   27  {\ldots}         0    0
2   Email 3    0   0    1    0    0   0    8    0    0  {\ldots}         0    0
3   Email 4    0   5   22    0    5   1   51    2   10  {\ldots}         0    0
4   Email 5    7   6   17    1    5   2   57    0    9  {\ldots}         0    0

   valued  lay  infrastructure  military  allowing  ff  dry  Prediction
0       0    0               0         0         0   0    0           0
1       0    0               0         0         0   1    0           0
2       0    0               0         0         0   0    0           0
3       0    0               0         0         0   0    0           0
4       0    0               0         0         0   1    0           0

[5 rows x 3002 columns]
    \end{Verbatim}

    \begin{tcolorbox}[breakable, size=fbox, boxrule=1pt, pad at break*=1mm,colback=cellbackground, colframe=cellborder]
\prompt{In}{incolor}{5}{\boxspacing}
\begin{Verbatim}[commandchars=\\\{\}]
\PY{c+c1}{\PYZsh{} Shape of the dataframe}
\PY{n+nb}{print}\PY{p}{(}\PY{n}{emails}\PY{o}{.}\PY{n}{shape}\PY{p}{)}
\end{Verbatim}
\end{tcolorbox}

    \begin{Verbatim}[commandchars=\\\{\}]
(5000, 3002)
    \end{Verbatim}

    \begin{tcolorbox}[breakable, size=fbox, boxrule=1pt, pad at break*=1mm,colback=cellbackground, colframe=cellborder]
\prompt{In}{incolor}{6}{\boxspacing}
\begin{Verbatim}[commandchars=\\\{\}]
\PY{c+c1}{\PYZsh{} Drop Email No. column}
\PY{n}{emails} \PY{o}{=} \PY{n}{emails}\PY{o}{.}\PY{n}{drop}\PY{p}{(}\PY{l+s+s1}{\PYZsq{}}\PY{l+s+s1}{Email No.}\PY{l+s+s1}{\PYZsq{}}\PY{p}{,} \PY{n}{axis}\PY{o}{=}\PY{l+m+mi}{1}\PY{p}{)}
\end{Verbatim}
\end{tcolorbox}

    \begin{tcolorbox}[breakable, size=fbox, boxrule=1pt, pad at break*=1mm,colback=cellbackground, colframe=cellborder]
\prompt{In}{incolor}{7}{\boxspacing}
\begin{Verbatim}[commandchars=\\\{\}]
\PY{c+c1}{\PYZsh{} Shape of the dataframe}
\PY{n+nb}{print}\PY{p}{(}\PY{n}{emails}\PY{o}{.}\PY{n}{shape}\PY{p}{)}
\end{Verbatim}
\end{tcolorbox}

    \begin{Verbatim}[commandchars=\\\{\}]
(5000, 3001)
    \end{Verbatim}

    For a single training/test set split, use Email 1-4000 as the training
set, Email 4001-5000 as the test set.

For 5-fold cross validation, split dataset in the following way.

\begin{itemize}
\tightlist
\item
  Fold 1, test set: Email 1-1000, training set: the rest (Email
  1001-5000)
\item
  Fold 2, test set: Email 1000-2000, training set: the rest
\item
  Fold 3, test set: Email 2000-3000, training set: the rest
\item
  Fold 4, test set: Email 3000-4000, training set: the rest
\item
  Fold 5, test set: Email 4000-5000, training set: the rest
\end{itemize}

    \begin{enumerate}
\def\labelenumi{\arabic{enumi}.}
\setcounter{enumi}{1}
\tightlist
\item
  (8 pts) Implement 1NN, Run 5-fold cross validation. Report accuracy,
  precision, and recall in each fold.
\end{enumerate}

    \begin{tcolorbox}[breakable, size=fbox, boxrule=1pt, pad at break*=1mm,colback=cellbackground, colframe=cellborder]
\prompt{In}{incolor}{45}{\boxspacing}
\begin{Verbatim}[commandchars=\\\{\}]
\PY{c+c1}{\PYZsh{} Implement 1NN, with 5\PYZhy{}fold cross validation without using sklearn}

\PY{c+c1}{\PYZsh{} Assuming you have a DataFrame `data` loaded from \PYZsq{}emails.csv\PYZsq{}}
\PY{c+c1}{\PYZsh{} X \PYZhy{} feature matrix, y \PYZhy{} labels}
\PY{n}{X} \PY{o}{=} \PY{n}{emails}\PY{o}{.}\PY{n}{drop}\PY{p}{(}\PY{n}{columns}\PY{o}{=}\PY{p}{[}\PY{l+s+s1}{\PYZsq{}}\PY{l+s+s1}{Prediction}\PY{l+s+s1}{\PYZsq{}}\PY{p}{]}\PY{p}{)}\PY{o}{.}\PY{n}{values}
\PY{n}{y} \PY{o}{=} \PY{n}{emails}\PY{p}{[}\PY{l+s+s1}{\PYZsq{}}\PY{l+s+s1}{Prediction}\PY{l+s+s1}{\PYZsq{}}\PY{p}{]}\PY{o}{.}\PY{n}{values}

\PY{c+c1}{\PYZsh{} Euclidean distance calculation}
\PY{k}{def} \PY{n+nf}{euclidean\PYZus{}distance}\PY{p}{(}\PY{n}{a}\PY{p}{,} \PY{n}{b}\PY{p}{)}\PY{p}{:}
    \PY{k}{return} \PY{n}{np}\PY{o}{.}\PY{n}{sqrt}\PY{p}{(}\PY{n}{np}\PY{o}{.}\PY{n}{sum}\PY{p}{(}\PY{p}{(}\PY{n}{a} \PY{o}{\PYZhy{}} \PY{n}{b}\PY{p}{)} \PY{o}{*}\PY{o}{*} \PY{l+m+mi}{2}\PY{p}{,} \PY{n}{axis}\PY{o}{=}\PY{l+m+mi}{1}\PY{p}{)}\PY{p}{)}

\PY{c+c1}{\PYZsh{} Prediction function for 1NN}
\PY{k}{def} \PY{n+nf}{predict\PYZus{}one}\PY{p}{(}\PY{n}{x\PYZus{}train}\PY{p}{,} \PY{n}{y\PYZus{}train}\PY{p}{,} \PY{n}{x\PYZus{}test\PYZus{}single}\PY{p}{)}\PY{p}{:}
    \PY{n}{distances} \PY{o}{=} \PY{n}{euclidean\PYZus{}distance}\PY{p}{(}\PY{n}{x\PYZus{}train}\PY{p}{,} \PY{n}{x\PYZus{}test\PYZus{}single}\PY{p}{)}
    \PY{n}{nearest\PYZus{}y} \PY{o}{=} \PY{n}{y\PYZus{}train}\PY{p}{[}\PY{n}{np}\PY{o}{.}\PY{n}{argmin}\PY{p}{(}\PY{n}{distances}\PY{p}{)}\PY{p}{]}
    \PY{k}{return} \PY{n}{nearest\PYZus{}y}

\PY{c+c1}{\PYZsh{} Metric calculations}
\PY{k}{def} \PY{n+nf}{compute\PYZus{}metrics}\PY{p}{(}\PY{n}{y\PYZus{}true}\PY{p}{,} \PY{n}{y\PYZus{}pred}\PY{p}{)}\PY{p}{:}
    \PY{n}{tp} \PY{o}{=} \PY{n}{np}\PY{o}{.}\PY{n}{sum}\PY{p}{(}\PY{p}{(}\PY{n}{y\PYZus{}true} \PY{o}{==} \PY{l+m+mi}{1}\PY{p}{)} \PY{o}{\PYZam{}} \PY{p}{(}\PY{n}{y\PYZus{}pred} \PY{o}{==} \PY{l+m+mi}{1}\PY{p}{)}\PY{p}{)}
    \PY{n}{tn} \PY{o}{=} \PY{n}{np}\PY{o}{.}\PY{n}{sum}\PY{p}{(}\PY{p}{(}\PY{n}{y\PYZus{}true} \PY{o}{==} \PY{l+m+mi}{0}\PY{p}{)} \PY{o}{\PYZam{}} \PY{p}{(}\PY{n}{y\PYZus{}pred} \PY{o}{==} \PY{l+m+mi}{0}\PY{p}{)}\PY{p}{)}
    \PY{n}{fp} \PY{o}{=} \PY{n}{np}\PY{o}{.}\PY{n}{sum}\PY{p}{(}\PY{p}{(}\PY{n}{y\PYZus{}true} \PY{o}{==} \PY{l+m+mi}{0}\PY{p}{)} \PY{o}{\PYZam{}} \PY{p}{(}\PY{n}{y\PYZus{}pred} \PY{o}{==} \PY{l+m+mi}{1}\PY{p}{)}\PY{p}{)}
    \PY{n}{fn} \PY{o}{=} \PY{n}{np}\PY{o}{.}\PY{n}{sum}\PY{p}{(}\PY{p}{(}\PY{n}{y\PYZus{}true} \PY{o}{==} \PY{l+m+mi}{1}\PY{p}{)} \PY{o}{\PYZam{}} \PY{p}{(}\PY{n}{y\PYZus{}pred} \PY{o}{==} \PY{l+m+mi}{0}\PY{p}{)}\PY{p}{)}
    
    \PY{n}{accuracy} \PY{o}{=} \PY{p}{(}\PY{n}{tp} \PY{o}{+} \PY{n}{tn}\PY{p}{)} \PY{o}{/} \PY{p}{(}\PY{n}{tp} \PY{o}{+} \PY{n}{tn} \PY{o}{+} \PY{n}{fp} \PY{o}{+} \PY{n}{fn}\PY{p}{)}
    \PY{n}{precision} \PY{o}{=} \PY{n}{tp} \PY{o}{/} \PY{p}{(}\PY{n}{tp} \PY{o}{+} \PY{n}{fp}\PY{p}{)} \PY{k}{if} \PY{n}{tp} \PY{o}{+} \PY{n}{fp} \PY{o}{!=} \PY{l+m+mi}{0} \PY{k}{else} \PY{l+m+mi}{0}
    \PY{n}{recall} \PY{o}{=} \PY{n}{tp} \PY{o}{/} \PY{p}{(}\PY{n}{tp} \PY{o}{+} \PY{n}{fn}\PY{p}{)} \PY{k}{if} \PY{n}{tp} \PY{o}{+} \PY{n}{fn} \PY{o}{!=} \PY{l+m+mi}{0} \PY{k}{else} \PY{l+m+mi}{0}
    
    \PY{k}{return} \PY{n}{accuracy}\PY{p}{,} \PY{n}{precision}\PY{p}{,} \PY{n}{recall}

\PY{c+c1}{\PYZsh{} 5\PYZhy{}fold CV splits}
\PY{n}{folds} \PY{o}{=} \PY{p}{[}
    \PY{p}{(}\PY{n}{np}\PY{o}{.}\PY{n}{arange}\PY{p}{(}\PY{l+m+mi}{0}\PY{p}{,} \PY{l+m+mi}{1000}\PY{p}{)}\PY{p}{,} \PY{n}{np}\PY{o}{.}\PY{n}{arange}\PY{p}{(}\PY{l+m+mi}{1000}\PY{p}{,} \PY{l+m+mi}{5000}\PY{p}{)}\PY{p}{)}\PY{p}{,}
    \PY{p}{(}\PY{n}{np}\PY{o}{.}\PY{n}{arange}\PY{p}{(}\PY{l+m+mi}{1000}\PY{p}{,} \PY{l+m+mi}{2000}\PY{p}{)}\PY{p}{,} \PY{n}{np}\PY{o}{.}\PY{n}{concatenate}\PY{p}{(}\PY{p}{[}\PY{n}{np}\PY{o}{.}\PY{n}{arange}\PY{p}{(}\PY{l+m+mi}{0}\PY{p}{,} \PY{l+m+mi}{1000}\PY{p}{)}\PY{p}{,} \PY{n}{np}\PY{o}{.}\PY{n}{arange}\PY{p}{(}\PY{l+m+mi}{2000}\PY{p}{,} \PY{l+m+mi}{5000}\PY{p}{)}\PY{p}{]}\PY{p}{)}\PY{p}{)}\PY{p}{,}
    \PY{p}{(}\PY{n}{np}\PY{o}{.}\PY{n}{arange}\PY{p}{(}\PY{l+m+mi}{2000}\PY{p}{,} \PY{l+m+mi}{3000}\PY{p}{)}\PY{p}{,} \PY{n}{np}\PY{o}{.}\PY{n}{concatenate}\PY{p}{(}\PY{p}{[}\PY{n}{np}\PY{o}{.}\PY{n}{arange}\PY{p}{(}\PY{l+m+mi}{0}\PY{p}{,} \PY{l+m+mi}{2000}\PY{p}{)}\PY{p}{,} \PY{n}{np}\PY{o}{.}\PY{n}{arange}\PY{p}{(}\PY{l+m+mi}{3000}\PY{p}{,} \PY{l+m+mi}{5000}\PY{p}{)}\PY{p}{]}\PY{p}{)}\PY{p}{)}\PY{p}{,}
    \PY{p}{(}\PY{n}{np}\PY{o}{.}\PY{n}{arange}\PY{p}{(}\PY{l+m+mi}{3000}\PY{p}{,} \PY{l+m+mi}{4000}\PY{p}{)}\PY{p}{,} \PY{n}{np}\PY{o}{.}\PY{n}{concatenate}\PY{p}{(}\PY{p}{[}\PY{n}{np}\PY{o}{.}\PY{n}{arange}\PY{p}{(}\PY{l+m+mi}{0}\PY{p}{,} \PY{l+m+mi}{3000}\PY{p}{)}\PY{p}{,} \PY{n}{np}\PY{o}{.}\PY{n}{arange}\PY{p}{(}\PY{l+m+mi}{4000}\PY{p}{,} \PY{l+m+mi}{5000}\PY{p}{)}\PY{p}{]}\PY{p}{)}\PY{p}{)}\PY{p}{,}
    \PY{p}{(}\PY{n}{np}\PY{o}{.}\PY{n}{arange}\PY{p}{(}\PY{l+m+mi}{4000}\PY{p}{,} \PY{l+m+mi}{5000}\PY{p}{)}\PY{p}{,} \PY{n}{np}\PY{o}{.}\PY{n}{arange}\PY{p}{(}\PY{l+m+mi}{0}\PY{p}{,} \PY{l+m+mi}{4000}\PY{p}{)}\PY{p}{)}
\PY{p}{]}

\PY{c+c1}{\PYZsh{} Applying 1NN and Evaluation}
\PY{k}{for} \PY{n}{i}\PY{p}{,} \PY{p}{(}\PY{n}{test\PYZus{}idx}\PY{p}{,} \PY{n}{train\PYZus{}idx}\PY{p}{)} \PY{o+ow}{in} \PY{n+nb}{enumerate}\PY{p}{(}\PY{n}{folds}\PY{p}{)}\PY{p}{:}
    \PY{n}{X\PYZus{}train}\PY{p}{,} \PY{n}{y\PYZus{}train} \PY{o}{=} \PY{n}{X}\PY{p}{[}\PY{n}{train\PYZus{}idx}\PY{p}{]}\PY{p}{,} \PY{n}{y}\PY{p}{[}\PY{n}{train\PYZus{}idx}\PY{p}{]}
    \PY{n}{X\PYZus{}test}\PY{p}{,} \PY{n}{y\PYZus{}test} \PY{o}{=} \PY{n}{X}\PY{p}{[}\PY{n}{test\PYZus{}idx}\PY{p}{]}\PY{p}{,} \PY{n}{y}\PY{p}{[}\PY{n}{test\PYZus{}idx}\PY{p}{]}
    
    \PY{n}{predictions} \PY{o}{=} \PY{n}{np}\PY{o}{.}\PY{n}{array}\PY{p}{(}\PY{p}{[}\PY{n}{predict\PYZus{}one}\PY{p}{(}\PY{n}{X\PYZus{}train}\PY{p}{,} \PY{n}{y\PYZus{}train}\PY{p}{,} \PY{n}{x\PYZus{}t}\PY{p}{)} \PY{k}{for} \PY{n}{x\PYZus{}t} \PY{o+ow}{in} \PY{n}{X\PYZus{}test}\PY{p}{]}\PY{p}{)}
    
    \PY{n}{acc}\PY{p}{,} \PY{n}{precision}\PY{p}{,} \PY{n}{recall} \PY{o}{=} \PY{n}{compute\PYZus{}metrics}\PY{p}{(}\PY{n}{y\PYZus{}test}\PY{p}{,} \PY{n}{predictions}\PY{p}{)}
    
    \PY{n+nb}{print}\PY{p}{(}\PY{l+s+sa}{f}\PY{l+s+s1}{\PYZsq{}}\PY{l+s+s1}{Fold }\PY{l+s+si}{\PYZob{}}\PY{n}{i}\PY{o}{+}\PY{l+m+mi}{1}\PY{l+s+si}{\PYZcb{}}\PY{l+s+s1}{:}\PY{l+s+s1}{\PYZsq{}}\PY{p}{)}
    \PY{n+nb}{print}\PY{p}{(}\PY{l+s+sa}{f}\PY{l+s+s1}{\PYZsq{}}\PY{l+s+s1}{Accuracy: }\PY{l+s+si}{\PYZob{}}\PY{n}{acc}\PY{l+s+si}{:}\PY{l+s+s1}{.4f}\PY{l+s+si}{\PYZcb{}}\PY{l+s+s1}{, Precision: }\PY{l+s+si}{\PYZob{}}\PY{n}{precision}\PY{l+s+si}{:}\PY{l+s+s1}{.4f}\PY{l+s+si}{\PYZcb{}}\PY{l+s+s1}{, Recall: }\PY{l+s+si}{\PYZob{}}\PY{n}{recall}\PY{l+s+si}{:}\PY{l+s+s1}{.4f}\PY{l+s+si}{\PYZcb{}}\PY{l+s+se}{\PYZbs{}n}\PY{l+s+s1}{\PYZsq{}}\PY{p}{)}
\end{Verbatim}
\end{tcolorbox}

    \begin{Verbatim}[commandchars=\\\{\}]
Fold 1:
Accuracy: 0.8250, Precision: 0.6545, Recall: 0.8175

Fold 2:
Accuracy: 0.8530, Precision: 0.6857, Recall: 0.8664

Fold 3:
Accuracy: 0.8620, Precision: 0.7212, Recall: 0.8380

Fold 4:
Accuracy: 0.8510, Precision: 0.7164, Recall: 0.8163

Fold 5:
Accuracy: 0.7750, Precision: 0.6057, Recall: 0.7582

    \end{Verbatim}

    (12 pts) Implement logistic regression (from scratch). Use gradient
descent (refer to question 6 from part 1) to find the optimal
parameters. You may need to tune your learning rate to find a good
optimum. Run 5-fold cross validation. Report accuracy, precision, and
recall in each fold.

    \begin{tcolorbox}[breakable, size=fbox, boxrule=1pt, pad at break*=1mm,colback=cellbackground, colframe=cellborder]
\prompt{In}{incolor}{49}{\boxspacing}
\begin{Verbatim}[commandchars=\\\{\}]
\PY{k}{def} \PY{n+nf}{sigmoid}\PY{p}{(}\PY{n}{z}\PY{p}{)}\PY{p}{:}
    \PY{c+c1}{\PYZsh{} Stabilization}
    \PY{n}{z} \PY{o}{=} \PY{n}{np}\PY{o}{.}\PY{n}{clip}\PY{p}{(}\PY{n}{z}\PY{p}{,} \PY{o}{\PYZhy{}}\PY{l+m+mi}{250}\PY{p}{,} \PY{l+m+mi}{250}\PY{p}{)}
    \PY{k}{return} \PY{l+m+mi}{1} \PY{o}{/} \PY{p}{(}\PY{l+m+mi}{1} \PY{o}{+} \PY{n}{np}\PY{o}{.}\PY{n}{exp}\PY{p}{(}\PY{o}{\PYZhy{}}\PY{n}{z}\PY{p}{)}\PY{p}{)}

\PY{k}{def} \PY{n+nf}{cost\PYZus{}function}\PY{p}{(}\PY{n}{y}\PY{p}{,} \PY{n}{y\PYZus{}pred}\PY{p}{)}\PY{p}{:}
    \PY{n}{m} \PY{o}{=} \PY{n+nb}{len}\PY{p}{(}\PY{n}{y}\PY{p}{)}
    \PY{k}{return} \PY{p}{(}\PY{o}{\PYZhy{}}\PY{l+m+mi}{1}\PY{o}{/}\PY{n}{m}\PY{p}{)} \PY{o}{*} \PY{n}{np}\PY{o}{.}\PY{n}{sum}\PY{p}{(}\PY{n}{y} \PY{o}{*} \PY{n}{np}\PY{o}{.}\PY{n}{log}\PY{p}{(}\PY{n}{y\PYZus{}pred}\PY{p}{)} \PY{o}{+} \PY{p}{(}\PY{l+m+mi}{1} \PY{o}{\PYZhy{}} \PY{n}{y}\PY{p}{)} \PY{o}{*} \PY{n}{np}\PY{o}{.}\PY{n}{log}\PY{p}{(}\PY{l+m+mi}{1} \PY{o}{\PYZhy{}} \PY{n}{y\PYZus{}pred}\PY{p}{)}\PY{p}{)}

\PY{k}{def} \PY{n+nf}{gradient\PYZus{}descent}\PY{p}{(}\PY{n}{X}\PY{p}{,} \PY{n}{y}\PY{p}{,} \PY{n}{theta}\PY{p}{,} \PY{n}{alpha}\PY{p}{,} \PY{n}{iterations}\PY{p}{)}\PY{p}{:}
    \PY{n}{m} \PY{o}{=} \PY{n+nb}{len}\PY{p}{(}\PY{n}{y}\PY{p}{)}
    \PY{k}{for} \PY{n}{\PYZus{}} \PY{o+ow}{in} \PY{n+nb}{range}\PY{p}{(}\PY{n}{iterations}\PY{p}{)}\PY{p}{:}
        \PY{n}{predictions} \PY{o}{=} \PY{n}{sigmoid}\PY{p}{(}\PY{n}{np}\PY{o}{.}\PY{n}{dot}\PY{p}{(}\PY{n}{X}\PY{p}{,} \PY{n}{theta}\PY{p}{)}\PY{p}{)}
        \PY{n}{errors} \PY{o}{=} \PY{n}{y} \PY{o}{\PYZhy{}} \PY{n}{predictions}
        \PY{n}{gradient} \PY{o}{=} \PY{n}{np}\PY{o}{.}\PY{n}{dot}\PY{p}{(}\PY{n}{X}\PY{o}{.}\PY{n}{T}\PY{p}{,} \PY{n}{errors}\PY{p}{)}
        \PY{n}{theta} \PY{o}{+}\PY{o}{=} \PY{n}{alpha} \PY{o}{*} \PY{n}{gradient}
    \PY{k}{return} \PY{n}{theta}

\PY{k}{def} \PY{n+nf}{predict}\PY{p}{(}\PY{n}{X}\PY{p}{,} \PY{n}{theta}\PY{p}{)}\PY{p}{:}
    \PY{n}{predictions} \PY{o}{=} \PY{n}{sigmoid}\PY{p}{(}\PY{n}{np}\PY{o}{.}\PY{n}{dot}\PY{p}{(}\PY{n}{X}\PY{p}{,} \PY{n}{theta}\PY{p}{)}\PY{p}{)}
    \PY{k}{return} \PY{p}{[}\PY{l+m+mi}{1} \PY{k}{if} \PY{n}{p} \PY{o}{\PYZgt{}}\PY{o}{=} \PY{l+m+mf}{0.5} \PY{k}{else} \PY{l+m+mi}{0} \PY{k}{for} \PY{n}{p} \PY{o+ow}{in} \PY{n}{predictions}\PY{p}{]}
\end{Verbatim}
\end{tcolorbox}

    \begin{tcolorbox}[breakable, size=fbox, boxrule=1pt, pad at break*=1mm,colback=cellbackground, colframe=cellborder]
\prompt{In}{incolor}{51}{\boxspacing}
\begin{Verbatim}[commandchars=\\\{\}]
\PY{c+c1}{\PYZsh{} Define thetas, alpha, and iterations}
\PY{n}{theta} \PY{o}{=} \PY{n}{np}\PY{o}{.}\PY{n}{zeros}\PY{p}{(}\PY{n}{X}\PY{o}{.}\PY{n}{shape}\PY{p}{[}\PY{l+m+mi}{1}\PY{p}{]}\PY{p}{)}
\PY{n}{alpha} \PY{o}{=} \PY{p}{[}\PY{l+m+mf}{0.1}\PY{p}{,} \PY{l+m+mf}{0.01}\PY{p}{,} \PY{l+m+mf}{0.001}\PY{p}{]}  \PY{c+c1}{\PYZsh{} Values of alpha to try}
\PY{n}{iterations} \PY{o}{=} \PY{l+m+mi}{1000}  

\PY{k}{def} \PY{n+nf}{compute\PYZus{}metrics}\PY{p}{(}\PY{n}{y\PYZus{}true}\PY{p}{,} \PY{n}{y\PYZus{}pred}\PY{p}{)}\PY{p}{:}
    \PY{c+c1}{\PYZsh{} True Positive, False Positive, True Negative and False Negative}
    \PY{n}{TP} \PY{o}{=} \PY{n}{np}\PY{o}{.}\PY{n}{sum}\PY{p}{(}\PY{p}{(}\PY{n}{np}\PY{o}{.}\PY{n}{array}\PY{p}{(}\PY{n}{y\PYZus{}true}\PY{p}{)} \PY{o}{==} \PY{l+m+mi}{1}\PY{p}{)} \PY{o}{\PYZam{}} \PY{p}{(}\PY{n}{np}\PY{o}{.}\PY{n}{array}\PY{p}{(}\PY{n}{y\PYZus{}pred}\PY{p}{)} \PY{o}{==} \PY{l+m+mi}{1}\PY{p}{)}\PY{p}{)}
    \PY{n}{FP} \PY{o}{=} \PY{n}{np}\PY{o}{.}\PY{n}{sum}\PY{p}{(}\PY{p}{(}\PY{n}{np}\PY{o}{.}\PY{n}{array}\PY{p}{(}\PY{n}{y\PYZus{}true}\PY{p}{)} \PY{o}{==} \PY{l+m+mi}{0}\PY{p}{)} \PY{o}{\PYZam{}} \PY{p}{(}\PY{n}{np}\PY{o}{.}\PY{n}{array}\PY{p}{(}\PY{n}{y\PYZus{}pred}\PY{p}{)} \PY{o}{==} \PY{l+m+mi}{1}\PY{p}{)}\PY{p}{)}
    \PY{n}{TN} \PY{o}{=} \PY{n}{np}\PY{o}{.}\PY{n}{sum}\PY{p}{(}\PY{p}{(}\PY{n}{np}\PY{o}{.}\PY{n}{array}\PY{p}{(}\PY{n}{y\PYZus{}true}\PY{p}{)} \PY{o}{==} \PY{l+m+mi}{0}\PY{p}{)} \PY{o}{\PYZam{}} \PY{p}{(}\PY{n}{np}\PY{o}{.}\PY{n}{array}\PY{p}{(}\PY{n}{y\PYZus{}pred}\PY{p}{)} \PY{o}{==} \PY{l+m+mi}{0}\PY{p}{)}\PY{p}{)}
    \PY{n}{FN} \PY{o}{=} \PY{n}{np}\PY{o}{.}\PY{n}{sum}\PY{p}{(}\PY{p}{(}\PY{n}{np}\PY{o}{.}\PY{n}{array}\PY{p}{(}\PY{n}{y\PYZus{}true}\PY{p}{)} \PY{o}{==} \PY{l+m+mi}{1}\PY{p}{)} \PY{o}{\PYZam{}} \PY{p}{(}\PY{n}{np}\PY{o}{.}\PY{n}{array}\PY{p}{(}\PY{n}{y\PYZus{}pred}\PY{p}{)} \PY{o}{==} \PY{l+m+mi}{0}\PY{p}{)}\PY{p}{)}

    \PY{c+c1}{\PYZsh{} Accuracy, Precision, and Recall}
    \PY{n}{acc} \PY{o}{=} \PY{p}{(}\PY{n}{TP} \PY{o}{+} \PY{n}{TN}\PY{p}{)} \PY{o}{/} \PY{p}{(}\PY{n}{TP} \PY{o}{+} \PY{n}{TN} \PY{o}{+} \PY{n}{FP} \PY{o}{+} \PY{n}{FN}\PY{p}{)}
    \PY{n}{precision} \PY{o}{=} \PY{n}{TP} \PY{o}{/} \PY{p}{(}\PY{n}{TP} \PY{o}{+} \PY{n}{FP}\PY{p}{)} \PY{k}{if} \PY{n}{TP} \PY{o}{+} \PY{n}{FP} \PY{o}{!=} \PY{l+m+mi}{0} \PY{k}{else} \PY{l+m+mi}{0}
    \PY{n}{recall} \PY{o}{=} \PY{n}{TP} \PY{o}{/} \PY{p}{(}\PY{n}{TP} \PY{o}{+} \PY{n}{FN}\PY{p}{)} \PY{k}{if} \PY{n}{TP} \PY{o}{+} \PY{n}{FN} \PY{o}{!=} \PY{l+m+mi}{0} \PY{k}{else} \PY{l+m+mi}{0}
    
    \PY{k}{return} \PY{n}{acc}\PY{p}{,} \PY{n}{precision}\PY{p}{,} \PY{n}{recall}
\end{Verbatim}
\end{tcolorbox}

    \begin{tcolorbox}[breakable, size=fbox, boxrule=1pt, pad at break*=1mm,colback=cellbackground, colframe=cellborder]
\prompt{In}{incolor}{52}{\boxspacing}
\begin{Verbatim}[commandchars=\\\{\}]
\PY{c+c1}{\PYZsh{} Applying Logistic Regression over 5\PYZhy{}fold CV}
\PY{k}{for} \PY{n}{i}\PY{p}{,} \PY{p}{(}\PY{n}{test\PYZus{}idx}\PY{p}{,} \PY{n}{train\PYZus{}idx}\PY{p}{)} \PY{o+ow}{in} \PY{n+nb}{enumerate}\PY{p}{(}\PY{n}{folds}\PY{p}{)}\PY{p}{:}
    \PY{n}{X\PYZus{}train}\PY{p}{,} \PY{n}{y\PYZus{}train} \PY{o}{=} \PY{n}{X}\PY{p}{[}\PY{n}{train\PYZus{}idx}\PY{p}{]}\PY{p}{,} \PY{n}{y}\PY{p}{[}\PY{n}{train\PYZus{}idx}\PY{p}{]}
    \PY{n}{X\PYZus{}test}\PY{p}{,} \PY{n}{y\PYZus{}test} \PY{o}{=} \PY{n}{X}\PY{p}{[}\PY{n}{test\PYZus{}idx}\PY{p}{]}\PY{p}{,} \PY{n}{y}\PY{p}{[}\PY{n}{test\PYZus{}idx}\PY{p}{]}
    
    \PY{k}{for} \PY{n}{a} \PY{o+ow}{in} \PY{n}{alpha}\PY{p}{:}
        \PY{n}{theta\PYZus{}optimal} \PY{o}{=} \PY{n}{gradient\PYZus{}descent}\PY{p}{(}\PY{n}{X\PYZus{}train}\PY{p}{,} \PY{n}{y\PYZus{}train}\PY{p}{,} \PY{n}{theta}\PY{p}{,} \PY{n}{a}\PY{p}{,} \PY{n}{iterations}\PY{p}{)}
        \PY{n}{predictions} \PY{o}{=} \PY{n}{predict}\PY{p}{(}\PY{n}{X\PYZus{}test}\PY{p}{,} \PY{n}{theta\PYZus{}optimal}\PY{p}{)}
        \PY{n}{acc}\PY{p}{,} \PY{n}{precision}\PY{p}{,} \PY{n}{recall} \PY{o}{=} \PY{n}{compute\PYZus{}metrics}\PY{p}{(}\PY{n}{y\PYZus{}test}\PY{p}{,} \PY{n}{predictions}\PY{p}{)}
        \PY{n+nb}{print}\PY{p}{(}\PY{l+s+sa}{f}\PY{l+s+s1}{\PYZsq{}}\PY{l+s+s1}{Fold }\PY{l+s+si}{\PYZob{}}\PY{n}{i}\PY{o}{+}\PY{l+m+mi}{1}\PY{l+s+si}{\PYZcb{}}\PY{l+s+s1}{, alpha=}\PY{l+s+si}{\PYZob{}}\PY{n}{a}\PY{l+s+si}{\PYZcb{}}\PY{l+s+s1}{:}\PY{l+s+s1}{\PYZsq{}}\PY{p}{)}
        \PY{n+nb}{print}\PY{p}{(}\PY{l+s+sa}{f}\PY{l+s+s1}{\PYZsq{}}\PY{l+s+s1}{Accuracy: }\PY{l+s+si}{\PYZob{}}\PY{n}{acc}\PY{l+s+si}{:}\PY{l+s+s1}{.4f}\PY{l+s+si}{\PYZcb{}}\PY{l+s+s1}{, Precision: }\PY{l+s+si}{\PYZob{}}\PY{n}{precision}\PY{l+s+si}{:}\PY{l+s+s1}{.4f}\PY{l+s+si}{\PYZcb{}}\PY{l+s+s1}{, Recall: }\PY{l+s+si}{\PYZob{}}\PY{n}{recall}\PY{l+s+si}{:}\PY{l+s+s1}{.4f}\PY{l+s+si}{\PYZcb{}}\PY{l+s+se}{\PYZbs{}n}\PY{l+s+s1}{\PYZsq{}}\PY{p}{)}
\end{Verbatim}
\end{tcolorbox}

    \begin{Verbatim}[commandchars=\\\{\}]
Fold 1, alpha=0.1:
Accuracy: 0.9060, Precision: 0.8687, Recall: 0.7895

Fold 1, alpha=0.01:
Accuracy: 0.9180, Precision: 0.8859, Recall: 0.8175

Fold 1, alpha=0.001:
Accuracy: 0.9180, Precision: 0.8859, Recall: 0.8175

Fold 2, alpha=0.1:
Accuracy: 0.9260, Precision: 0.8889, Recall: 0.8375

Fold 2, alpha=0.01:
Accuracy: 0.9300, Precision: 0.9027, Recall: 0.8375

Fold 2, alpha=0.001:
Accuracy: 0.9300, Precision: 0.9027, Recall: 0.8375

Fold 3, alpha=0.1:
Accuracy: 0.9250, Precision: 0.9409, Recall: 0.7852

Fold 3, alpha=0.01:
Accuracy: 0.9300, Precision: 0.9315, Recall: 0.8134

Fold 3, alpha=0.001:
Accuracy: 0.9300, Precision: 0.9315, Recall: 0.8134

Fold 4, alpha=0.1:
Accuracy: 0.9540, Precision: 0.9366, Recall: 0.9048

Fold 4, alpha=0.01:
Accuracy: 0.9540, Precision: 0.9366, Recall: 0.9048

Fold 4, alpha=0.001:
Accuracy: 0.9540, Precision: 0.9366, Recall: 0.9048

Fold 5, alpha=0.1:
Accuracy: 0.9430, Precision: 0.9029, Recall: 0.9118

Fold 5, alpha=0.01:
Accuracy: 0.9400, Precision: 0.8994, Recall: 0.9052

Fold 5, alpha=0.001:
Accuracy: 0.9410, Precision: 0.8997, Recall: 0.9085

    \end{Verbatim}

    \begin{tcolorbox}[breakable, size=fbox, boxrule=1pt, pad at break*=1mm,colback=cellbackground, colframe=cellborder]
\prompt{In}{incolor}{53}{\boxspacing}
\begin{Verbatim}[commandchars=\\\{\}]
\PY{n}{results\PYZus{}log} \PY{o}{=} \PY{p}{\PYZob{}}
    \PY{l+s+s2}{\PYZdq{}}\PY{l+s+s2}{Fold}\PY{l+s+s2}{\PYZdq{}}\PY{p}{:} \PY{p}{[}\PY{l+m+mi}{1}\PY{p}{,} \PY{l+m+mi}{1}\PY{p}{,} \PY{l+m+mi}{1}\PY{p}{,} \PY{l+m+mi}{2}\PY{p}{,} \PY{l+m+mi}{2}\PY{p}{,} \PY{l+m+mi}{2}\PY{p}{,} \PY{l+m+mi}{3}\PY{p}{,} \PY{l+m+mi}{3}\PY{p}{,} \PY{l+m+mi}{3}\PY{p}{,} \PY{l+m+mi}{4}\PY{p}{,} \PY{l+m+mi}{4}\PY{p}{,} \PY{l+m+mi}{4}\PY{p}{,} \PY{l+m+mi}{5}\PY{p}{,} \PY{l+m+mi}{5}\PY{p}{,} \PY{l+m+mi}{5}\PY{p}{]}\PY{p}{,}
    \PY{l+s+s2}{\PYZdq{}}\PY{l+s+s2}{Alpha}\PY{l+s+s2}{\PYZdq{}}\PY{p}{:} \PY{p}{[}\PY{l+m+mf}{0.1}\PY{p}{,} \PY{l+m+mf}{0.01}\PY{p}{,} \PY{l+m+mf}{0.001}\PY{p}{,} \PY{l+m+mf}{0.1}\PY{p}{,} \PY{l+m+mf}{0.01}\PY{p}{,} \PY{l+m+mf}{0.001}\PY{p}{,} \PY{l+m+mf}{0.1}\PY{p}{,} \PY{l+m+mf}{0.01}\PY{p}{,} \PY{l+m+mf}{0.001}\PY{p}{,} \PY{l+m+mf}{0.1}\PY{p}{,} \PY{l+m+mf}{0.01}\PY{p}{,} \PY{l+m+mf}{0.001}\PY{p}{,} \PY{l+m+mf}{0.1}\PY{p}{,} \PY{l+m+mf}{0.01}\PY{p}{,} \PY{l+m+mf}{0.001}\PY{p}{]}\PY{p}{,}
    \PY{l+s+s2}{\PYZdq{}}\PY{l+s+s2}{Accuracy}\PY{l+s+s2}{\PYZdq{}}\PY{p}{:} \PY{p}{[}\PY{l+m+mf}{0.906}\PY{p}{,} \PY{l+m+mf}{0.918}\PY{p}{,} \PY{l+m+mf}{0.918}\PY{p}{,} \PY{l+m+mf}{0.926}\PY{p}{,} \PY{l+m+mf}{0.930}\PY{p}{,} \PY{l+m+mf}{0.930}\PY{p}{,} \PY{l+m+mf}{0.925}\PY{p}{,} \PY{l+m+mf}{0.930}\PY{p}{,} \PY{l+m+mf}{0.930}\PY{p}{,} \PY{l+m+mf}{0.954}\PY{p}{,} \PY{l+m+mf}{0.954}\PY{p}{,} \PY{l+m+mf}{0.954}\PY{p}{,} \PY{l+m+mf}{0.943}\PY{p}{,} \PY{l+m+mf}{0.940}\PY{p}{,} \PY{l+m+mf}{0.941}\PY{p}{]}\PY{p}{,}
    \PY{l+s+s2}{\PYZdq{}}\PY{l+s+s2}{Precision}\PY{l+s+s2}{\PYZdq{}}\PY{p}{:} \PY{p}{[}\PY{l+m+mf}{0.8687}\PY{p}{,} \PY{l+m+mf}{0.8859}\PY{p}{,} \PY{l+m+mf}{0.8859}\PY{p}{,} \PY{l+m+mf}{0.8889}\PY{p}{,} \PY{l+m+mf}{0.9027}\PY{p}{,} \PY{l+m+mf}{0.9027}\PY{p}{,} \PY{l+m+mf}{0.9409}\PY{p}{,} \PY{l+m+mf}{0.9315}\PY{p}{,} \PY{l+m+mf}{0.9315}\PY{p}{,} \PY{l+m+mf}{0.9366}\PY{p}{,} \PY{l+m+mf}{0.9366}\PY{p}{,} \PY{l+m+mf}{0.9366}\PY{p}{,} \PY{l+m+mf}{0.9029}\PY{p}{,} \PY{l+m+mf}{0.8994}\PY{p}{,} \PY{l+m+mf}{0.8997}\PY{p}{]}\PY{p}{,}
    \PY{l+s+s2}{\PYZdq{}}\PY{l+s+s2}{Recall}\PY{l+s+s2}{\PYZdq{}}\PY{p}{:} \PY{p}{[}\PY{l+m+mf}{0.7895}\PY{p}{,} \PY{l+m+mf}{0.8175}\PY{p}{,} \PY{l+m+mf}{0.8175}\PY{p}{,} \PY{l+m+mf}{0.8375}\PY{p}{,} \PY{l+m+mf}{0.8375}\PY{p}{,} \PY{l+m+mf}{0.8375}\PY{p}{,} \PY{l+m+mf}{0.7852}\PY{p}{,} \PY{l+m+mf}{0.8134}\PY{p}{,} \PY{l+m+mf}{0.8134}\PY{p}{,} \PY{l+m+mf}{0.9048}\PY{p}{,} \PY{l+m+mf}{0.9048}\PY{p}{,} \PY{l+m+mf}{0.9048}\PY{p}{,} \PY{l+m+mf}{0.9118}\PY{p}{,} \PY{l+m+mf}{0.9052}\PY{p}{,} \PY{l+m+mf}{0.9085}\PY{p}{]}
\PY{p}{\PYZcb{}}

\PY{n}{results\PYZus{}log\PYZus{}df} \PY{o}{=} \PY{n}{pd}\PY{o}{.}\PY{n}{DataFrame}\PY{p}{(}\PY{n}{results\PYZus{}log}\PY{p}{)}
\end{Verbatim}
\end{tcolorbox}

    \begin{tcolorbox}[breakable, size=fbox, boxrule=1pt, pad at break*=1mm,colback=cellbackground, colframe=cellborder]
\prompt{In}{incolor}{55}{\boxspacing}
\begin{Verbatim}[commandchars=\\\{\}]
\PY{c+c1}{\PYZsh{} Ignore warnings}
\PY{k+kn}{import} \PY{n+nn}{warnings}
\PY{n}{warnings}\PY{o}{.}\PY{n}{filterwarnings}\PY{p}{(}\PY{l+s+s1}{\PYZsq{}}\PY{l+s+s1}{ignore}\PY{l+s+s1}{\PYZsq{}}\PY{p}{)}
\end{Verbatim}
\end{tcolorbox}

    \begin{tcolorbox}[breakable, size=fbox, boxrule=1pt, pad at break*=1mm,colback=cellbackground, colframe=cellborder]
\prompt{In}{incolor}{58}{\boxspacing}
\begin{Verbatim}[commandchars=\\\{\}]
\PY{c+c1}{\PYZsh{} Presenting the results by alpha}

\PY{c+c1}{\PYZsh{} Accuracy}
\PY{n}{plt}\PY{o}{.}\PY{n}{figure}\PY{p}{(}\PY{n}{figsize}\PY{o}{=}\PY{p}{(}\PY{l+m+mi}{4}\PY{p}{,} \PY{l+m+mi}{4}\PY{p}{)}\PY{p}{)}
\PY{n}{sns}\PY{o}{.}\PY{n}{barplot}\PY{p}{(}\PY{n}{x}\PY{o}{=}\PY{l+s+s1}{\PYZsq{}}\PY{l+s+s1}{Alpha}\PY{l+s+s1}{\PYZsq{}}\PY{p}{,} \PY{n}{y}\PY{o}{=}\PY{l+s+s1}{\PYZsq{}}\PY{l+s+s1}{Accuracy}\PY{l+s+s1}{\PYZsq{}}\PY{p}{,} \PY{n}{data}\PY{o}{=}\PY{n}{results\PYZus{}log\PYZus{}df}\PY{p}{)}
\PY{n}{plt}\PY{o}{.}\PY{n}{title}\PY{p}{(}\PY{l+s+s1}{\PYZsq{}}\PY{l+s+s1}{Accuracy by Alpha}\PY{l+s+s1}{\PYZsq{}}\PY{p}{)}
\PY{n}{plt}\PY{o}{.}\PY{n}{ylim}\PY{p}{(}\PY{p}{[}\PY{l+m+mf}{0.9}\PY{p}{,} \PY{l+m+mf}{0.96}\PY{p}{]}\PY{p}{)}
\PY{n}{plt}\PY{o}{.}\PY{n}{show}\PY{p}{(}\PY{p}{)}

\PY{c+c1}{\PYZsh{} Precision}
\PY{n}{plt}\PY{o}{.}\PY{n}{figure}\PY{p}{(}\PY{n}{figsize}\PY{o}{=}\PY{p}{(}\PY{l+m+mi}{4}\PY{p}{,} \PY{l+m+mi}{4}\PY{p}{)}\PY{p}{)}
\PY{n}{sns}\PY{o}{.}\PY{n}{barplot}\PY{p}{(}\PY{n}{x}\PY{o}{=}\PY{l+s+s1}{\PYZsq{}}\PY{l+s+s1}{Alpha}\PY{l+s+s1}{\PYZsq{}}\PY{p}{,} \PY{n}{y}\PY{o}{=}\PY{l+s+s1}{\PYZsq{}}\PY{l+s+s1}{Precision}\PY{l+s+s1}{\PYZsq{}}\PY{p}{,} \PY{n}{data}\PY{o}{=}\PY{n}{results\PYZus{}log\PYZus{}df}\PY{p}{)}
\PY{n}{plt}\PY{o}{.}\PY{n}{title}\PY{p}{(}\PY{l+s+s1}{\PYZsq{}}\PY{l+s+s1}{Precision by Alpha}\PY{l+s+s1}{\PYZsq{}}\PY{p}{)}
\PY{n}{plt}\PY{o}{.}\PY{n}{ylim}\PY{p}{(}\PY{p}{[}\PY{l+m+mf}{0.85}\PY{p}{,} \PY{l+m+mf}{0.95}\PY{p}{]}\PY{p}{)}
\PY{n}{plt}\PY{o}{.}\PY{n}{show}\PY{p}{(}\PY{p}{)}

\PY{c+c1}{\PYZsh{} Recall}
\PY{n}{plt}\PY{o}{.}\PY{n}{figure}\PY{p}{(}\PY{n}{figsize}\PY{o}{=}\PY{p}{(}\PY{l+m+mi}{4}\PY{p}{,} \PY{l+m+mi}{4}\PY{p}{)}\PY{p}{)}
\PY{n}{sns}\PY{o}{.}\PY{n}{barplot}\PY{p}{(}\PY{n}{x}\PY{o}{=}\PY{l+s+s1}{\PYZsq{}}\PY{l+s+s1}{Alpha}\PY{l+s+s1}{\PYZsq{}}\PY{p}{,} \PY{n}{y}\PY{o}{=}\PY{l+s+s1}{\PYZsq{}}\PY{l+s+s1}{Recall}\PY{l+s+s1}{\PYZsq{}}\PY{p}{,} \PY{n}{data}\PY{o}{=}\PY{n}{results\PYZus{}log\PYZus{}df}\PY{p}{)}
\PY{n}{plt}\PY{o}{.}\PY{n}{title}\PY{p}{(}\PY{l+s+s1}{\PYZsq{}}\PY{l+s+s1}{Recall by Alpha}\PY{l+s+s1}{\PYZsq{}}\PY{p}{)}
\PY{n}{plt}\PY{o}{.}\PY{n}{ylim}\PY{p}{(}\PY{p}{[}\PY{l+m+mf}{0.75}\PY{p}{,} \PY{l+m+mf}{0.95}\PY{p}{]}\PY{p}{)}
\PY{n}{plt}\PY{o}{.}\PY{n}{show}\PY{p}{(}\PY{p}{)}
\end{Verbatim}
\end{tcolorbox}

    \begin{center}
    \adjustimage{max size={0.9\linewidth}{0.9\paperheight}}{Homework 3 - Dario Placencio_files/Homework 3 - Dario Placencio_59_0.png}
    \end{center}
    { \hspace*{\fill} \\}
    
    \begin{center}
    \adjustimage{max size={0.9\linewidth}{0.9\paperheight}}{Homework 3 - Dario Placencio_files/Homework 3 - Dario Placencio_59_1.png}
    \end{center}
    { \hspace*{\fill} \\}
    
    \begin{center}
    \adjustimage{max size={0.9\linewidth}{0.9\paperheight}}{Homework 3 - Dario Placencio_files/Homework 3 - Dario Placencio_59_2.png}
    \end{center}
    { \hspace*{\fill} \\}
    
    \begin{tcolorbox}[breakable, size=fbox, boxrule=1pt, pad at break*=1mm,colback=cellbackground, colframe=cellborder]
\prompt{In}{incolor}{59}{\boxspacing}
\begin{Verbatim}[commandchars=\\\{\}]
\PY{c+c1}{\PYZsh{} Calculate the most optimal alpha}
\PY{n}{optimal\PYZus{}alpha} \PY{o}{=} \PY{n}{results\PYZus{}log\PYZus{}df}\PY{o}{.}\PY{n}{groupby}\PY{p}{(}\PY{l+s+s1}{\PYZsq{}}\PY{l+s+s1}{Alpha}\PY{l+s+s1}{\PYZsq{}}\PY{p}{)}\PY{o}{.}\PY{n}{mean}\PY{p}{(}\PY{p}{)}\PY{o}{.}\PY{n}{idxmax}\PY{p}{(}\PY{p}{)}\PY{o}{.}\PY{n}{values}\PY{p}{[}\PY{l+m+mi}{0}\PY{p}{]}
\PY{n+nb}{print}\PY{p}{(}\PY{l+s+sa}{f}\PY{l+s+s1}{\PYZsq{}}\PY{l+s+s1}{The most optimal alpha is }\PY{l+s+si}{\PYZob{}}\PY{n}{optimal\PYZus{}alpha}\PY{l+s+si}{\PYZcb{}}\PY{l+s+s1}{\PYZsq{}}\PY{p}{)}
\end{Verbatim}
\end{tcolorbox}

    \begin{Verbatim}[commandchars=\\\{\}]
The most optimal alpha is 0.001
    \end{Verbatim}

    \begin{tcolorbox}[breakable, size=fbox, boxrule=1pt, pad at break*=1mm,colback=cellbackground, colframe=cellborder]
\prompt{In}{incolor}{62}{\boxspacing}
\begin{Verbatim}[commandchars=\\\{\}]
\PY{c+c1}{\PYZsh{} Report accuracy, precision, and recall for the optimal alpha for each fold }
\PY{n}{results\PYZus{}log\PYZus{}df}\PY{p}{[}\PY{n}{results\PYZus{}log\PYZus{}df}\PY{p}{[}\PY{l+s+s1}{\PYZsq{}}\PY{l+s+s1}{Alpha}\PY{l+s+s1}{\PYZsq{}}\PY{p}{]} \PY{o}{==} \PY{n}{optimal\PYZus{}alpha}\PY{p}{]}
\end{Verbatim}
\end{tcolorbox}

            \begin{tcolorbox}[breakable, size=fbox, boxrule=.5pt, pad at break*=1mm, opacityfill=0]
\prompt{Out}{outcolor}{62}{\boxspacing}
\begin{Verbatim}[commandchars=\\\{\}]
    Fold  Alpha  Accuracy  Precision  Recall
2      1  0.001     0.918     0.8859  0.8175
5      2  0.001     0.930     0.9027  0.8375
8      3  0.001     0.930     0.9315  0.8134
11     4  0.001     0.954     0.9366  0.9048
14     5  0.001     0.941     0.8997  0.9085
\end{Verbatim}
\end{tcolorbox}
        
    \begin{enumerate}
\def\labelenumi{\arabic{enumi}.}
\setcounter{enumi}{3}
\tightlist
\item
  (10 pts) Run 5-fold cross validation with kNN varying k (k=1, 3, 5, 7,
  10). Plot the average accuracy versus k, and list the average accuracy
  of each case.
\end{enumerate}

    \begin{tcolorbox}[breakable, size=fbox, boxrule=1pt, pad at break*=1mm,colback=cellbackground, colframe=cellborder]
\prompt{In}{incolor}{65}{\boxspacing}
\begin{Verbatim}[commandchars=\\\{\}]
\PY{k}{def} \PY{n+nf}{euclidean\PYZus{}distance}\PY{p}{(}\PY{n}{x1}\PY{p}{,} \PY{n}{x2}\PY{p}{)}\PY{p}{:}
    \PY{k}{return} \PY{n}{np}\PY{o}{.}\PY{n}{sqrt}\PY{p}{(}\PY{n}{np}\PY{o}{.}\PY{n}{sum}\PY{p}{(}\PY{p}{(}\PY{n}{x1} \PY{o}{\PYZhy{}} \PY{n}{x2}\PY{p}{)}\PY{o}{*}\PY{o}{*}\PY{l+m+mi}{2}\PY{p}{,} \PY{n}{axis}\PY{o}{=}\PY{l+m+mi}{1}\PY{p}{)}\PY{p}{)}

\PY{k}{def} \PY{n+nf}{kNN}\PY{p}{(}\PY{n}{X\PYZus{}train}\PY{p}{,} \PY{n}{y\PYZus{}train}\PY{p}{,} \PY{n}{X\PYZus{}test}\PY{p}{,} \PY{n}{k}\PY{p}{)}\PY{p}{:}
    \PY{n}{y\PYZus{}pred} \PY{o}{=} \PY{n}{np}\PY{o}{.}\PY{n}{zeros}\PY{p}{(}\PY{n}{X\PYZus{}test}\PY{o}{.}\PY{n}{shape}\PY{p}{[}\PY{l+m+mi}{0}\PY{p}{]}\PY{p}{)}
    \PY{k}{for} \PY{n}{idx}\PY{p}{,} \PY{n}{test\PYZus{}instance} \PY{o+ow}{in} \PY{n+nb}{enumerate}\PY{p}{(}\PY{n}{X\PYZus{}test}\PY{p}{)}\PY{p}{:}
        \PY{n}{distances} \PY{o}{=} \PY{n}{euclidean\PYZus{}distance}\PY{p}{(}\PY{n}{X\PYZus{}train}\PY{p}{,} \PY{n}{test\PYZus{}instance}\PY{p}{)}
        \PY{n}{neighbors\PYZus{}indices} \PY{o}{=} \PY{n}{np}\PY{o}{.}\PY{n}{argsort}\PY{p}{(}\PY{n}{distances}\PY{p}{)}\PY{p}{[}\PY{p}{:}\PY{n}{k}\PY{p}{]}
        \PY{n}{neighbors\PYZus{}labels} \PY{o}{=} \PY{n}{y\PYZus{}train}\PY{p}{[}\PY{n}{neighbors\PYZus{}indices}\PY{p}{]}
        \PY{n}{y\PYZus{}pred}\PY{p}{[}\PY{n}{idx}\PY{p}{]} \PY{o}{=} \PY{n}{np}\PY{o}{.}\PY{n}{bincount}\PY{p}{(}\PY{n}{neighbors\PYZus{}labels}\PY{p}{)}\PY{o}{.}\PY{n}{argmax}\PY{p}{(}\PY{p}{)}  \PY{c+c1}{\PYZsh{} majority vote}
    \PY{k}{return} \PY{n}{y\PYZus{}pred}
\end{Verbatim}
\end{tcolorbox}

    \begin{tcolorbox}[breakable, size=fbox, boxrule=1pt, pad at break*=1mm,colback=cellbackground, colframe=cellborder]
\prompt{In}{incolor}{68}{\boxspacing}
\begin{Verbatim}[commandchars=\\\{\}]
\PY{k}{def} \PY{n+nf}{cross\PYZus{}validation\PYZus{}split}\PY{p}{(}\PY{n}{X}\PY{p}{,} \PY{n}{y}\PY{p}{,} \PY{n}{n\PYZus{}folds}\PY{o}{=}\PY{l+m+mi}{5}\PY{p}{)}\PY{p}{:}
    \PY{n}{fold\PYZus{}sizes} \PY{o}{=} \PY{n}{X}\PY{o}{.}\PY{n}{shape}\PY{p}{[}\PY{l+m+mi}{0}\PY{p}{]} \PY{o}{/}\PY{o}{/} \PY{n}{n\PYZus{}folds}
    \PY{n}{indices} \PY{o}{=} \PY{n}{np}\PY{o}{.}\PY{n}{arange}\PY{p}{(}\PY{n}{X}\PY{o}{.}\PY{n}{shape}\PY{p}{[}\PY{l+m+mi}{0}\PY{p}{]}\PY{p}{)}
    \PY{n}{np}\PY{o}{.}\PY{n}{random}\PY{o}{.}\PY{n}{shuffle}\PY{p}{(}\PY{n}{indices}\PY{p}{)}
    
    \PY{n}{folds} \PY{o}{=} \PY{p}{[}\PY{p}{]}
    \PY{k}{for} \PY{n}{i} \PY{o+ow}{in} \PY{n+nb}{range}\PY{p}{(}\PY{n}{n\PYZus{}folds}\PY{p}{)}\PY{p}{:}
        \PY{n}{test\PYZus{}idx} \PY{o}{=} \PY{n}{indices}\PY{p}{[}\PY{n}{i}\PY{o}{*}\PY{n}{fold\PYZus{}sizes}\PY{p}{:}\PY{p}{(}\PY{n}{i}\PY{o}{+}\PY{l+m+mi}{1}\PY{p}{)}\PY{o}{*}\PY{n}{fold\PYZus{}sizes}\PY{p}{]}
        \PY{n}{train\PYZus{}idx} \PY{o}{=} \PY{n}{np}\PY{o}{.}\PY{n}{setdiff1d}\PY{p}{(}\PY{n}{indices}\PY{p}{,} \PY{n}{test\PYZus{}idx}\PY{p}{)}
        \PY{n}{folds}\PY{o}{.}\PY{n}{append}\PY{p}{(}\PY{p}{(}\PY{n}{train\PYZus{}idx}\PY{p}{,} \PY{n}{test\PYZus{}idx}\PY{p}{)}\PY{p}{)}
    \PY{k}{return} \PY{n}{folds}
\end{Verbatim}
\end{tcolorbox}

    \begin{tcolorbox}[breakable, size=fbox, boxrule=1pt, pad at break*=1mm,colback=cellbackground, colframe=cellborder]
\prompt{In}{incolor}{66}{\boxspacing}
\begin{Verbatim}[commandchars=\\\{\}]
\PY{k}{def} \PY{n+nf}{accuracy}\PY{p}{(}\PY{n}{y\PYZus{}true}\PY{p}{,} \PY{n}{y\PYZus{}pred}\PY{p}{)}\PY{p}{:}
    \PY{k}{return} \PY{n}{np}\PY{o}{.}\PY{n}{mean}\PY{p}{(}\PY{n}{y\PYZus{}true} \PY{o}{==} \PY{n}{y\PYZus{}pred}\PY{p}{)}
\end{Verbatim}
\end{tcolorbox}

    \begin{tcolorbox}[breakable, size=fbox, boxrule=1pt, pad at break*=1mm,colback=cellbackground, colframe=cellborder]
\prompt{In}{incolor}{70}{\boxspacing}
\begin{Verbatim}[commandchars=\\\{\}]
\PY{k}{def} \PY{n+nf}{perform\PYZus{}knn\PYZus{}cv}\PY{p}{(}\PY{n}{X}\PY{p}{,} \PY{n}{y}\PY{p}{,} \PY{n}{k\PYZus{}values}\PY{p}{,} \PY{n}{folds}\PY{p}{)}\PY{p}{:}
    \PY{n}{results} \PY{o}{=} \PY{n}{np}\PY{o}{.}\PY{n}{zeros}\PY{p}{(}\PY{p}{(}\PY{n+nb}{len}\PY{p}{(}\PY{n}{k\PYZus{}values}\PY{p}{)}\PY{p}{,} \PY{n+nb}{len}\PY{p}{(}\PY{n}{folds}\PY{p}{)}\PY{p}{)}\PY{p}{)}
    
    \PY{k}{for} \PY{n}{i}\PY{p}{,} \PY{n}{k} \PY{o+ow}{in} \PY{n+nb}{enumerate}\PY{p}{(}\PY{n}{k\PYZus{}values}\PY{p}{)}\PY{p}{:}
        \PY{k}{for} \PY{n}{j}\PY{p}{,} \PY{p}{(}\PY{n}{test\PYZus{}idx}\PY{p}{,} \PY{n}{train\PYZus{}idx}\PY{p}{)} \PY{o+ow}{in} \PY{n+nb}{enumerate}\PY{p}{(}\PY{n}{folds}\PY{p}{)}\PY{p}{:}
            \PY{n}{X\PYZus{}train}\PY{p}{,} \PY{n}{y\PYZus{}train} \PY{o}{=} \PY{n}{X}\PY{p}{[}\PY{n}{train\PYZus{}idx}\PY{p}{]}\PY{p}{,} \PY{n}{y}\PY{p}{[}\PY{n}{train\PYZus{}idx}\PY{p}{]}
            \PY{n}{X\PYZus{}test}\PY{p}{,} \PY{n}{y\PYZus{}test} \PY{o}{=} \PY{n}{X}\PY{p}{[}\PY{n}{test\PYZus{}idx}\PY{p}{]}\PY{p}{,} \PY{n}{y}\PY{p}{[}\PY{n}{test\PYZus{}idx}\PY{p}{]}
            
            \PY{n}{y\PYZus{}pred} \PY{o}{=} \PY{n}{kNN}\PY{p}{(}\PY{n}{X\PYZus{}train}\PY{p}{,} \PY{n}{y\PYZus{}train}\PY{p}{,} \PY{n}{X\PYZus{}test}\PY{p}{,} \PY{n}{k}\PY{p}{)}
            \PY{n}{acc} \PY{o}{=} \PY{n}{accuracy}\PY{p}{(}\PY{n}{y\PYZus{}test}\PY{p}{,} \PY{n}{y\PYZus{}pred}\PY{p}{)}
            \PY{n}{results}\PY{p}{[}\PY{n}{i}\PY{p}{,} \PY{n}{j}\PY{p}{]} \PY{o}{=} \PY{n}{acc}
            
    \PY{k}{return} \PY{n}{results}\PY{o}{.}\PY{n}{mean}\PY{p}{(}\PY{n}{axis}\PY{o}{=}\PY{l+m+mi}{1}\PY{p}{)}
\end{Verbatim}
\end{tcolorbox}

    \begin{tcolorbox}[breakable, size=fbox, boxrule=1pt, pad at break*=1mm,colback=cellbackground, colframe=cellborder]
\prompt{In}{incolor}{71}{\boxspacing}
\begin{Verbatim}[commandchars=\\\{\}]
\PY{n}{k\PYZus{}values} \PY{o}{=} \PY{p}{[}\PY{l+m+mi}{1}\PY{p}{,} \PY{l+m+mi}{3}\PY{p}{,} \PY{l+m+mi}{5}\PY{p}{,} \PY{l+m+mi}{7}\PY{p}{,} \PY{l+m+mi}{10}\PY{p}{]}
\PY{n}{avg\PYZus{}accuracies} \PY{o}{=} \PY{n}{perform\PYZus{}knn\PYZus{}cv}\PY{p}{(}\PY{n}{X}\PY{p}{,} \PY{n}{y}\PY{p}{,} \PY{n}{k\PYZus{}values}\PY{p}{,} \PY{n}{folds}\PY{p}{)}

\PY{n}{plt}\PY{o}{.}\PY{n}{plot}\PY{p}{(}\PY{n}{k\PYZus{}values}\PY{p}{,} \PY{n}{avg\PYZus{}accuracies}\PY{p}{,} \PY{n}{marker}\PY{o}{=}\PY{l+s+s1}{\PYZsq{}}\PY{l+s+s1}{o}\PY{l+s+s1}{\PYZsq{}}\PY{p}{)}
\PY{n}{plt}\PY{o}{.}\PY{n}{title}\PY{p}{(}\PY{l+s+s1}{\PYZsq{}}\PY{l+s+s1}{Average accuracy vs. k}\PY{l+s+s1}{\PYZsq{}}\PY{p}{)}
\PY{n}{plt}\PY{o}{.}\PY{n}{xlabel}\PY{p}{(}\PY{l+s+s1}{\PYZsq{}}\PY{l+s+s1}{k}\PY{l+s+s1}{\PYZsq{}}\PY{p}{)}
\PY{n}{plt}\PY{o}{.}\PY{n}{ylabel}\PY{p}{(}\PY{l+s+s1}{\PYZsq{}}\PY{l+s+s1}{Average accuracy}\PY{l+s+s1}{\PYZsq{}}\PY{p}{)}
\PY{n}{plt}\PY{o}{.}\PY{n}{grid}\PY{p}{(}\PY{k+kc}{True}\PY{p}{)}
\PY{n}{plt}\PY{o}{.}\PY{n}{show}\PY{p}{(}\PY{p}{)}\PY{p}{;}
\end{Verbatim}
\end{tcolorbox}

    \begin{center}
    \adjustimage{max size={0.9\linewidth}{0.9\paperheight}}{Homework 3 - Dario Placencio_files/Homework 3 - Dario Placencio_67_0.png}
    \end{center}
    { \hspace*{\fill} \\}
    
    \begin{tcolorbox}[breakable, size=fbox, boxrule=1pt, pad at break*=1mm,colback=cellbackground, colframe=cellborder]
\prompt{In}{incolor}{72}{\boxspacing}
\begin{Verbatim}[commandchars=\\\{\}]
\PY{c+c1}{\PYZsh{} Printing the average accuracies}
\PY{k}{for} \PY{n}{k}\PY{p}{,} \PY{n}{acc} \PY{o+ow}{in} \PY{n+nb}{zip}\PY{p}{(}\PY{n}{k\PYZus{}values}\PY{p}{,} \PY{n}{avg\PYZus{}accuracies}\PY{p}{)}\PY{p}{:}
    \PY{n+nb}{print}\PY{p}{(}\PY{l+s+sa}{f}\PY{l+s+s2}{\PYZdq{}}\PY{l+s+s2}{Average accuracy for k=}\PY{l+s+si}{\PYZob{}}\PY{n}{k}\PY{l+s+si}{\PYZcb{}}\PY{l+s+s2}{: }\PY{l+s+si}{\PYZob{}}\PY{n}{acc}\PY{l+s+si}{:}\PY{l+s+s2}{.4f}\PY{l+s+si}{\PYZcb{}}\PY{l+s+s2}{\PYZdq{}}\PY{p}{)}
\end{Verbatim}
\end{tcolorbox}

    \begin{Verbatim}[commandchars=\\\{\}]
Average accuracy for k=1: 0.8344
Average accuracy for k=3: 0.8410
Average accuracy for k=5: 0.8418
Average accuracy for k=7: 0.8452
Average accuracy for k=10: 0.8558
    \end{Verbatim}

    \begin{enumerate}
\def\labelenumi{\arabic{enumi}.}
\setcounter{enumi}{4}
\tightlist
\item
  (10 pts) Use a single training/test setting. Train kNN (k=5) and
  logistic regression on the training set, and draw ROC curves based on
  the test set.
\end{enumerate}

    \begin{tcolorbox}[breakable, size=fbox, boxrule=1pt, pad at break*=1mm,colback=cellbackground, colframe=cellborder]
\prompt{In}{incolor}{8}{\boxspacing}
\begin{Verbatim}[commandchars=\\\{\}]
\PY{c+c1}{\PYZsh{} X \PYZhy{} feature matrix, y \PYZhy{} labels}
\PY{n}{X} \PY{o}{=} \PY{n}{emails}\PY{o}{.}\PY{n}{drop}\PY{p}{(}\PY{n}{columns}\PY{o}{=}\PY{p}{[}\PY{l+s+s1}{\PYZsq{}}\PY{l+s+s1}{Prediction}\PY{l+s+s1}{\PYZsq{}}\PY{p}{]}\PY{p}{)}\PY{o}{.}\PY{n}{values}
\PY{n}{y} \PY{o}{=} \PY{n}{emails}\PY{p}{[}\PY{l+s+s1}{\PYZsq{}}\PY{l+s+s1}{Prediction}\PY{l+s+s1}{\PYZsq{}}\PY{p}{]}\PY{o}{.}\PY{n}{values}
\end{Verbatim}
\end{tcolorbox}

    \begin{tcolorbox}[breakable, size=fbox, boxrule=1pt, pad at break*=1mm,colback=cellbackground, colframe=cellborder]
\prompt{In}{incolor}{20}{\boxspacing}
\begin{Verbatim}[commandchars=\\\{\}]
\PY{c+c1}{\PYZsh{} Train Test Split}
\PY{n}{X\PYZus{}train}\PY{p}{,} \PY{n}{y\PYZus{}train} \PY{o}{=} \PY{n}{X}\PY{p}{[}\PY{p}{:}\PY{l+m+mi}{4000}\PY{p}{]}\PY{p}{,} \PY{n}{y}\PY{p}{[}\PY{p}{:}\PY{l+m+mi}{4000}\PY{p}{]}
\PY{n}{X\PYZus{}test}\PY{p}{,} \PY{n}{y\PYZus{}test} \PY{o}{=} \PY{n}{X}\PY{p}{[}\PY{l+m+mi}{4000}\PY{p}{:}\PY{p}{]}\PY{p}{,} \PY{n}{y}\PY{p}{[}\PY{l+m+mi}{4000}\PY{p}{:}\PY{p}{]}
\end{Verbatim}
\end{tcolorbox}

    \begin{tcolorbox}[breakable, size=fbox, boxrule=1pt, pad at break*=1mm,colback=cellbackground, colframe=cellborder]
\prompt{In}{incolor}{28}{\boxspacing}
\begin{Verbatim}[commandchars=\\\{\}]
\PY{k}{def} \PY{n+nf}{euclidean\PYZus{}distance}\PY{p}{(}\PY{n}{x1}\PY{p}{,} \PY{n}{x2}\PY{p}{)}\PY{p}{:}
    \PY{k}{return} \PY{n}{np}\PY{o}{.}\PY{n}{sqrt}\PY{p}{(}\PY{n}{np}\PY{o}{.}\PY{n}{sum}\PY{p}{(}\PY{p}{(}\PY{n}{x1} \PY{o}{\PYZhy{}} \PY{n}{x2}\PY{p}{)} \PY{o}{*}\PY{o}{*} \PY{l+m+mi}{2}\PY{p}{,} \PY{n}{axis}\PY{o}{=}\PY{l+m+mi}{1}\PY{p}{)}\PY{p}{)}

\PY{k}{def} \PY{n+nf}{kNN\PYZus{}proba}\PY{p}{(}\PY{n}{X\PYZus{}train}\PY{p}{,} \PY{n}{y\PYZus{}train}\PY{p}{,} \PY{n}{X\PYZus{}test}\PY{p}{,} \PY{n}{k}\PY{o}{=}\PY{l+m+mi}{5}\PY{p}{)}\PY{p}{:}
    \PY{n}{y\PYZus{}pred\PYZus{}proba} \PY{o}{=} \PY{p}{[}\PY{p}{]}
    \PY{k}{for} \PY{n}{x} \PY{o+ow}{in} \PY{n}{X\PYZus{}test}\PY{p}{:}
        \PY{n}{distances} \PY{o}{=} \PY{n}{euclidean\PYZus{}distance}\PY{p}{(}\PY{n}{X\PYZus{}train}\PY{p}{,} \PY{n}{x}\PY{p}{)}
        \PY{n}{k\PYZus{}neighbors} \PY{o}{=} \PY{n}{y\PYZus{}train}\PY{p}{[}\PY{n}{np}\PY{o}{.}\PY{n}{argsort}\PY{p}{(}\PY{n}{distances}\PY{p}{)}\PY{p}{[}\PY{p}{:}\PY{n}{k}\PY{p}{]}\PY{p}{]}
        \PY{n}{proba} \PY{o}{=} \PY{n}{np}\PY{o}{.}\PY{n}{sum}\PY{p}{(}\PY{n}{k\PYZus{}neighbors}\PY{p}{)} \PY{o}{/} \PY{n}{k}  \PY{c+c1}{\PYZsh{} Summing 1s and dividing by k gives probability for class 1}
        \PY{n}{y\PYZus{}pred\PYZus{}proba}\PY{o}{.}\PY{n}{append}\PY{p}{(}\PY{n}{proba}\PY{p}{)}
    \PY{k}{return} \PY{n}{np}\PY{o}{.}\PY{n}{array}\PY{p}{(}\PY{n}{y\PYZus{}pred\PYZus{}proba}\PY{p}{)}
\end{Verbatim}
\end{tcolorbox}

    \begin{tcolorbox}[breakable, size=fbox, boxrule=1pt, pad at break*=1mm,colback=cellbackground, colframe=cellborder]
\prompt{In}{incolor}{33}{\boxspacing}
\begin{Verbatim}[commandchars=\\\{\}]
\PY{c+c1}{\PYZsh{} Calculate the probabilities}
\PY{n}{y\PYZus{}pred\PYZus{}proba\PYZus{}5nn} \PY{o}{=} \PY{n}{kNN\PYZus{}proba}\PY{p}{(}\PY{n}{X\PYZus{}train}\PY{p}{,} \PY{n}{y\PYZus{}train}\PY{p}{,} \PY{n}{X\PYZus{}test}\PY{p}{,} \PY{n}{k}\PY{o}{=}\PY{l+m+mi}{5}\PY{p}{)}
\end{Verbatim}
\end{tcolorbox}

    \begin{tcolorbox}[breakable, size=fbox, boxrule=1pt, pad at break*=1mm,colback=cellbackground, colframe=cellborder]
\prompt{In}{incolor}{30}{\boxspacing}
\begin{Verbatim}[commandchars=\\\{\}]
\PY{k}{def} \PY{n+nf}{sigmoid}\PY{p}{(}\PY{n}{z}\PY{p}{)}\PY{p}{:}
    \PY{c+c1}{\PYZsh{} Stabilization}
    \PY{n}{z} \PY{o}{=} \PY{n}{np}\PY{o}{.}\PY{n}{clip}\PY{p}{(}\PY{n}{z}\PY{p}{,} \PY{o}{\PYZhy{}}\PY{l+m+mi}{250}\PY{p}{,} \PY{l+m+mi}{250}\PY{p}{)}
    \PY{k}{return} \PY{l+m+mi}{1} \PY{o}{/} \PY{p}{(}\PY{l+m+mi}{1} \PY{o}{+} \PY{n}{np}\PY{o}{.}\PY{n}{exp}\PY{p}{(}\PY{o}{\PYZhy{}}\PY{n}{z}\PY{p}{)}\PY{p}{)}

\PY{k}{def} \PY{n+nf}{compute\PYZus{}cost}\PY{p}{(}\PY{n}{X}\PY{p}{,} \PY{n}{y}\PY{p}{,} \PY{n}{theta}\PY{p}{)}\PY{p}{:}
    \PY{n}{m} \PY{o}{=} \PY{n+nb}{len}\PY{p}{(}\PY{n}{y}\PY{p}{)}
    \PY{n}{h} \PY{o}{=} \PY{n}{sigmoid}\PY{p}{(}\PY{n}{np}\PY{o}{.}\PY{n}{dot}\PY{p}{(}\PY{n}{X}\PY{p}{,} \PY{n}{theta}\PY{p}{)}\PY{p}{)}
    \PY{k}{return} \PY{p}{(}\PY{o}{\PYZhy{}}\PY{l+m+mi}{1}\PY{o}{/}\PY{n}{m}\PY{p}{)} \PY{o}{*} \PY{n}{np}\PY{o}{.}\PY{n}{sum}\PY{p}{(}\PY{n}{y}\PY{o}{*}\PY{n}{np}\PY{o}{.}\PY{n}{log}\PY{p}{(}\PY{n}{h}\PY{p}{)} \PY{o}{+} \PY{p}{(}\PY{l+m+mi}{1}\PY{o}{\PYZhy{}}\PY{n}{y}\PY{p}{)}\PY{o}{*}\PY{n}{np}\PY{o}{.}\PY{n}{log}\PY{p}{(}\PY{l+m+mi}{1}\PY{o}{\PYZhy{}}\PY{n}{h}\PY{p}{)}\PY{p}{)}

\PY{k}{def} \PY{n+nf}{gradient\PYZus{}descent}\PY{p}{(}\PY{n}{X}\PY{p}{,} \PY{n}{y}\PY{p}{,} \PY{n}{theta}\PY{p}{,} \PY{n}{alpha}\PY{p}{,} \PY{n}{num\PYZus{}iters}\PY{p}{)}\PY{p}{:}
    \PY{n}{m} \PY{o}{=} \PY{n+nb}{len}\PY{p}{(}\PY{n}{y}\PY{p}{)}
    \PY{k}{for} \PY{n}{\PYZus{}} \PY{o+ow}{in} \PY{n+nb}{range}\PY{p}{(}\PY{n}{num\PYZus{}iters}\PY{p}{)}\PY{p}{:}
        \PY{n}{h} \PY{o}{=} \PY{n}{sigmoid}\PY{p}{(}\PY{n}{np}\PY{o}{.}\PY{n}{dot}\PY{p}{(}\PY{n}{X}\PY{p}{,} \PY{n}{theta}\PY{p}{)}\PY{p}{)}
        \PY{n}{gradient} \PY{o}{=} \PY{n}{np}\PY{o}{.}\PY{n}{dot}\PY{p}{(}\PY{n}{X}\PY{o}{.}\PY{n}{T}\PY{p}{,} \PY{p}{(}\PY{n}{h} \PY{o}{\PYZhy{}} \PY{n}{y}\PY{p}{)}\PY{p}{)} \PY{o}{/} \PY{n}{m}
        \PY{n}{theta} \PY{o}{=} \PY{n}{theta} \PY{o}{\PYZhy{}} \PY{n}{alpha} \PY{o}{*} \PY{n}{gradient}
    \PY{k}{return} \PY{n}{theta}

\PY{c+c1}{\PYZsh{} Initialize theta}
\PY{n}{theta} \PY{o}{=} \PY{n}{np}\PY{o}{.}\PY{n}{zeros}\PY{p}{(}\PY{n}{X\PYZus{}train}\PY{o}{.}\PY{n}{shape}\PY{p}{[}\PY{l+m+mi}{1}\PY{p}{]}\PY{p}{)}

\PY{c+c1}{\PYZsh{} Set learning rate (alpha) and number of iterations}
\PY{n}{alpha} \PY{o}{=} \PY{l+m+mf}{0.01}
\PY{n}{num\PYZus{}iters} \PY{o}{=} \PY{l+m+mi}{500}

\PY{c+c1}{\PYZsh{} Train the model}
\PY{n}{theta} \PY{o}{=} \PY{n}{gradient\PYZus{}descent}\PY{p}{(}\PY{n}{X\PYZus{}train}\PY{p}{,} \PY{n}{y\PYZus{}train}\PY{p}{,} \PY{n}{theta}\PY{p}{,} \PY{n}{alpha}\PY{p}{,} \PY{n}{num\PYZus{}iters}\PY{p}{)}
\end{Verbatim}
\end{tcolorbox}

    \begin{tcolorbox}[breakable, size=fbox, boxrule=1pt, pad at break*=1mm,colback=cellbackground, colframe=cellborder]
\prompt{In}{incolor}{35}{\boxspacing}
\begin{Verbatim}[commandchars=\\\{\}]
\PY{c+c1}{\PYZsh{} Predict probabilities on the test set}
\PY{n}{y\PYZus{}pred\PYZus{}proba\PYZus{}log} \PY{o}{=} \PY{n}{sigmoid}\PY{p}{(}\PY{n}{np}\PY{o}{.}\PY{n}{dot}\PY{p}{(}\PY{n}{X\PYZus{}test}\PY{p}{,} \PY{n}{theta}\PY{p}{)}\PY{p}{)}
\end{Verbatim}
\end{tcolorbox}

    \begin{tcolorbox}[breakable, size=fbox, boxrule=1pt, pad at break*=1mm,colback=cellbackground, colframe=cellborder]
\prompt{In}{incolor}{32}{\boxspacing}
\begin{Verbatim}[commandchars=\\\{\}]
\PY{k}{def} \PY{n+nf}{roc\PYZus{}curve}\PY{p}{(}\PY{n}{y\PYZus{}true}\PY{p}{,} \PY{n}{y\PYZus{}pred\PYZus{}proba}\PY{p}{)}\PY{p}{:}
    \PY{n}{thresholds} \PY{o}{=} \PY{n}{np}\PY{o}{.}\PY{n}{arange}\PY{p}{(}\PY{l+m+mi}{0}\PY{p}{,} \PY{l+m+mf}{1.01}\PY{p}{,} \PY{l+m+mf}{0.01}\PY{p}{)}
    \PY{n}{tpr\PYZus{}list} \PY{o}{=} \PY{p}{[}\PY{p}{]}
    \PY{n}{fpr\PYZus{}list} \PY{o}{=} \PY{p}{[}\PY{p}{]}
    
    \PY{k}{for} \PY{n}{threshold} \PY{o+ow}{in} \PY{n}{thresholds}\PY{p}{:}
        \PY{c+c1}{\PYZsh{} Convert predicted probabilities to binary predictions based on the threshold}
        \PY{n}{pred\PYZus{}class} \PY{o}{=} \PY{n}{np}\PY{o}{.}\PY{n}{array}\PY{p}{(}\PY{p}{[}\PY{l+m+mi}{1} \PY{k}{if} \PY{n}{x} \PY{o}{\PYZgt{}}\PY{o}{=} \PY{n}{threshold} \PY{k}{else} \PY{l+m+mi}{0} \PY{k}{for} \PY{n}{x} \PY{o+ow}{in} \PY{n}{y\PYZus{}pred\PYZus{}proba}\PY{p}{]}\PY{p}{)}
        
        \PY{c+c1}{\PYZsh{} Calculate True Positives (TP), False Positives (FP), True Negatives (TN), and False Negatives (FN)}
        \PY{n}{tp} \PY{o}{=} \PY{n}{np}\PY{o}{.}\PY{n}{sum}\PY{p}{(}\PY{p}{(}\PY{n}{y\PYZus{}true} \PY{o}{==} \PY{l+m+mi}{1}\PY{p}{)} \PY{o}{\PYZam{}} \PY{p}{(}\PY{n}{pred\PYZus{}class} \PY{o}{==} \PY{l+m+mi}{1}\PY{p}{)}\PY{p}{)}
        \PY{n}{fp} \PY{o}{=} \PY{n}{np}\PY{o}{.}\PY{n}{sum}\PY{p}{(}\PY{p}{(}\PY{n}{y\PYZus{}true} \PY{o}{==} \PY{l+m+mi}{0}\PY{p}{)} \PY{o}{\PYZam{}} \PY{p}{(}\PY{n}{pred\PYZus{}class} \PY{o}{==} \PY{l+m+mi}{1}\PY{p}{)}\PY{p}{)}
        \PY{n}{tn} \PY{o}{=} \PY{n}{np}\PY{o}{.}\PY{n}{sum}\PY{p}{(}\PY{p}{(}\PY{n}{y\PYZus{}true} \PY{o}{==} \PY{l+m+mi}{0}\PY{p}{)} \PY{o}{\PYZam{}} \PY{p}{(}\PY{n}{pred\PYZus{}class} \PY{o}{==} \PY{l+m+mi}{0}\PY{p}{)}\PY{p}{)}
        \PY{n}{fn} \PY{o}{=} \PY{n}{np}\PY{o}{.}\PY{n}{sum}\PY{p}{(}\PY{p}{(}\PY{n}{y\PYZus{}true} \PY{o}{==} \PY{l+m+mi}{1}\PY{p}{)} \PY{o}{\PYZam{}} \PY{p}{(}\PY{n}{pred\PYZus{}class} \PY{o}{==} \PY{l+m+mi}{0}\PY{p}{)}\PY{p}{)}
        
        \PY{c+c1}{\PYZsh{} Calculate True Positive Rate (TPR) and False Positive Rate (FPR)}
        \PY{n}{tpr} \PY{o}{=} \PY{n}{tp} \PY{o}{/} \PY{p}{(}\PY{n}{tp} \PY{o}{+} \PY{n}{fn}\PY{p}{)} \PY{k}{if} \PY{p}{(}\PY{n}{tp} \PY{o}{+} \PY{n}{fn}\PY{p}{)} \PY{o}{!=} \PY{l+m+mi}{0} \PY{k}{else} \PY{l+m+mi}{0}
        \PY{n}{fpr} \PY{o}{=} \PY{n}{fp} \PY{o}{/} \PY{p}{(}\PY{n}{fp} \PY{o}{+} \PY{n}{tn}\PY{p}{)} \PY{k}{if} \PY{p}{(}\PY{n}{fp} \PY{o}{+} \PY{n}{tn}\PY{p}{)} \PY{o}{!=} \PY{l+m+mi}{0} \PY{k}{else} \PY{l+m+mi}{0}
        
        \PY{n}{tpr\PYZus{}list}\PY{o}{.}\PY{n}{append}\PY{p}{(}\PY{n}{tpr}\PY{p}{)}
        \PY{n}{fpr\PYZus{}list}\PY{o}{.}\PY{n}{append}\PY{p}{(}\PY{n}{fpr}\PY{p}{)}
    
    \PY{k}{return} \PY{n}{fpr\PYZus{}list}\PY{p}{,} \PY{n}{tpr\PYZus{}list}
\end{Verbatim}
\end{tcolorbox}

    \begin{tcolorbox}[breakable, size=fbox, boxrule=1pt, pad at break*=1mm,colback=cellbackground, colframe=cellborder]
\prompt{In}{incolor}{37}{\boxspacing}
\begin{Verbatim}[commandchars=\\\{\}]
\PY{c+c1}{\PYZsh{} ROC}
\PY{n}{fpr\PYZus{}5nn}\PY{p}{,} \PY{n}{tpr\PYZus{}5nn} \PY{o}{=} \PY{n}{roc\PYZus{}curve}\PY{p}{(}\PY{n}{y\PYZus{}test}\PY{p}{,} \PY{n}{y\PYZus{}pred\PYZus{}proba\PYZus{}5nn}\PY{p}{)}
\PY{n}{fpr\PYZus{}log}\PY{p}{,} \PY{n}{tpr\PYZus{}log} \PY{o}{=} \PY{n}{roc\PYZus{}curve}\PY{p}{(}\PY{n}{y\PYZus{}test}\PY{p}{,} \PY{n}{y\PYZus{}pred\PYZus{}proba\PYZus{}log}\PY{p}{)}
\end{Verbatim}
\end{tcolorbox}

    \begin{tcolorbox}[breakable, size=fbox, boxrule=1pt, pad at break*=1mm,colback=cellbackground, colframe=cellborder]
\prompt{In}{incolor}{39}{\boxspacing}
\begin{Verbatim}[commandchars=\\\{\}]
\PY{c+c1}{\PYZsh{} Plotting}
\PY{n}{plt}\PY{o}{.}\PY{n}{figure}\PY{p}{(}\PY{n}{figsize}\PY{o}{=}\PY{p}{(}\PY{l+m+mi}{8}\PY{p}{,} \PY{l+m+mi}{8}\PY{p}{)}\PY{p}{)}
\PY{n}{plt}\PY{o}{.}\PY{n}{plot}\PY{p}{(}\PY{n}{fpr\PYZus{}5nn}\PY{p}{,} \PY{n}{tpr\PYZus{}5nn}\PY{p}{,} \PY{n}{label}\PY{o}{=}\PY{l+s+s1}{\PYZsq{}}\PY{l+s+s1}{k\PYZhy{}NN (k=5)}\PY{l+s+s1}{\PYZsq{}}\PY{p}{)}
\PY{n}{plt}\PY{o}{.}\PY{n}{plot}\PY{p}{(}\PY{n}{fpr\PYZus{}log}\PY{p}{,} \PY{n}{tpr\PYZus{}log}\PY{p}{,} \PY{n}{label}\PY{o}{=}\PY{l+s+s1}{\PYZsq{}}\PY{l+s+s1}{Logistic Regression}\PY{l+s+s1}{\PYZsq{}}\PY{p}{)}
\PY{n}{plt}\PY{o}{.}\PY{n}{plot}\PY{p}{(}\PY{p}{[}\PY{l+m+mi}{0}\PY{p}{,} \PY{l+m+mi}{1}\PY{p}{]}\PY{p}{,} \PY{p}{[}\PY{l+m+mi}{0}\PY{p}{,} \PY{l+m+mi}{1}\PY{p}{]}\PY{p}{,} \PY{n}{linestyle}\PY{o}{=}\PY{l+s+s1}{\PYZsq{}}\PY{l+s+s1}{\PYZhy{}\PYZhy{}}\PY{l+s+s1}{\PYZsq{}}\PY{p}{,} \PY{n}{color}\PY{o}{=}\PY{l+s+s1}{\PYZsq{}}\PY{l+s+s1}{gray}\PY{l+s+s1}{\PYZsq{}}\PY{p}{)}
\PY{n}{plt}\PY{o}{.}\PY{n}{title}\PY{p}{(}\PY{l+s+s1}{\PYZsq{}}\PY{l+s+s1}{ROC Curve}\PY{l+s+s1}{\PYZsq{}}\PY{p}{)}
\PY{n}{plt}\PY{o}{.}\PY{n}{xlabel}\PY{p}{(}\PY{l+s+s1}{\PYZsq{}}\PY{l+s+s1}{False Positive Rate}\PY{l+s+s1}{\PYZsq{}}\PY{p}{)}
\PY{n}{plt}\PY{o}{.}\PY{n}{ylabel}\PY{p}{(}\PY{l+s+s1}{\PYZsq{}}\PY{l+s+s1}{True Positive Rate}\PY{l+s+s1}{\PYZsq{}}\PY{p}{)}
\PY{n}{plt}\PY{o}{.}\PY{n}{xlim}\PY{p}{(}\PY{p}{[}\PY{l+m+mi}{0}\PY{p}{,} \PY{l+m+mi}{1}\PY{p}{]}\PY{p}{)}
\PY{n}{plt}\PY{o}{.}\PY{n}{ylim}\PY{p}{(}\PY{p}{[}\PY{l+m+mi}{0}\PY{p}{,} \PY{l+m+mi}{1}\PY{p}{]}\PY{p}{)}
\PY{n}{plt}\PY{o}{.}\PY{n}{grid}\PY{p}{(}\PY{k+kc}{True}\PY{p}{)}
\PY{n}{plt}\PY{o}{.}\PY{n}{legend}\PY{p}{(}\PY{p}{)}
\PY{n}{plt}\PY{o}{.}\PY{n}{show}\PY{p}{(}\PY{p}{)}\PY{p}{;}
\end{Verbatim}
\end{tcolorbox}

    \begin{center}
    \adjustimage{max size={0.9\linewidth}{0.9\paperheight}}{Homework 3 - Dario Placencio_files/Homework 3 - Dario Placencio_78_0.png}
    \end{center}
    { \hspace*{\fill} \\}
    

    % Add a bibliography block to the postdoc
    
    
    
\end{document}
